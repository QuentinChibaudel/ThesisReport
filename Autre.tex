\section{L'accès aux soins : un élément moteur de l'évolution du XX\textsuperscript{ième} siècle}

\subsection{La notion de concept}

Tout au long de ce rapport de thèse, plusieurs concepts seront définis avec rigueur et précision. Pour pouvoir apporter ces définitions, il faut s'interroger sur la notion de concept : quels sont les éléments qui composent un concept ? qu'est ce qui caractérise un concept ?\\
D'après Meleis (1991)\footnote{Meleis , A.I. (1991) \textit{Theoretical Nursing: Development and Progress}, 2\textsuperscript{nd} edn. Philadelphia, PA: Lippincott.}, un concept peut-être décrit en tant que mots clefs ou labels qui peuvent aider à résumer une liste de pensées. Il faut que tous les attributs apparaissent dans la formulation du concept.
Selon McKenna (1997)\footnote{Mckenna, H . (1997) \textit{Nursing Theories and Models.} London: Routledge.}, on peut considérer plusieurs types de concepts : 
\begin{itemize}
\item le concept modèle : scénario illustrant la bonne utilisation du concept, tous les attributs du concept doivent être présents
\item le concept contraire : scénario qui illustre la non considération du concept, une utilisation inapproprié
\item le faux concept : scénario qui illustre une mauvaise utilisation, une mauvaise considération du concept dans un contexte donné
\end{itemize}

Enfin, selon Walker \& Avant (1995)\footnote{Walker, L. O. \& Avant, K. C. (1995) \textit{Strategies for Theory Construction in Nursing.} Norwalk, CT: Appleton and Lange.}, des éléments supplémentaires peuvent aider à mieux cerner la notion de concept : 
\begin{itemize}
\item les antécédents c'est-à-dire les caractéristiques qui sont présentes avant l'apparition d'un concept
\item les conséquences : les caractéristiques suivant un phénomène et pour lequel les variables sont définies
\end{itemize}

En conclusion, un concept s'appuie sur : 
\begin{itemize}
\item une liste exhaustives d'éléments (labels ou mots clefs) descriptifs
\item une évolution dans le temps
\item un scénario d'illustration de la mise en place du concept, plus ou moins correcte
\end{itemize}

\begin{figure}[htbp]%insertion d'une image : définition d'un environnement figure
\begin{center}
\includegraphics[scale=0.45]{images/NotionConcept.png} 
\end{center}
\caption{Description de la notion de concept}
\label{Description de la notion de concept}
\end{figure}


\subsection*{L'espérance de vie au XX\textsuperscript{ième} siècle}
\subsubsection*{Définissons le concept d'espérance de vie}

Une définition courte et précise de l'espérance de vie a été donnée comme suit par une étude menée au Canada : \og nombre prévu d'années de vie qui restent, selon les conditions de mortalité actuelles.\fg (impossible de retrouver la référence ...) --> chercher autre chose \newline \newline
L'INSEE\footnote{http://www.insee.fr} définit l'espérance de vie à la naissance (ou à l'âge 0) comme la durée de vie moyenne - autrement dit l'âge moyen au décès - d'une génération fictive soumise aux conditions de mortalité de l'année (23 Juin 2016). Elle caractérise la mortalité indépendamment de la structure par âge.


C'est un cas particulier de l'espérance de vie à l'âge X. Cette espérance représente le nombre moyen d'années restant à vivre pour une génération fictive d'âge X qui aurait, à chaque âge, la probabilité de décéder observée cette année-là.
Autrement dit, c'est le nombre moyen d'années restant à vivre au-delà de cet âge X (ou durée de survie moyenne à l'âge X), dans les conditions de mortalité par âge de l'année considérée.\\
Les deux éléments communs qui ressortent de ces deux définitions de l'espérance de vie et qu'il faut retenir sont : 
\begin{itemize}
\item que l'espérance de vie est une durée prévisionnelle de vie pour une personne
\item cette durée est basée sur l'hypothèse que les conditions ne vont plus évoluer à partir du moment où le calcul est effectué
\item elle dépend des conditions de vie, sociales et géographiques de la personne 
\end{itemize} 

\subsubsection*{Constats et premiers éléments de réflexion}
Le XX\textsuperscript{ième} siècle, et notamment la période suivant la seconde guerre mondiale, aura été le théâtre d'un phénomène totalement nouveau : une augmentation sans précédent de l'espérance de vie. Ainsi, selon le rapport publié par l'Organisation Mondiale de la Santé (2014)\footnote{Organisation Mondiale de la Santé (2014) \textit{Statistiques sanitaires mondiales}}, une petite fille née en 2012 peut s'attendre à vivre en moyenne 72,7 ans et un petit garçon né la même année 68,1 ans. C'est six ans de plus que l'espérance de vie mondial pour un enfant né en 1990, 22 ans auparavant. 
Ce phénomène s'observe cependant de façon très hétérogène en fonction des pays et de leur richesse. Par exemple, dans un pays riche comme la France,  selon l'INSEE, en 1960, l'espérance de vie pour une femme était de 73,6 ans: pour un homme, elle était de 67 ans. En 2013, elle était respectivement de 85 ans et de 78,8 ans. L'espérance de vie a donc augmenté de 11,4 ans pour les femmes et de 11,8 ans pour les hommes en 53 ans. On peut donc en déduire qu'en France, en 1960, l'espérance de vie moyenne était de 70,3 ans et qu'elle était en 2013 de 81,9 soit une progression moyenne de 11,6 ans en 53 ans.
En comparaison, dans un pays pauvre d'Afrique comme le Sénégal, d'après la banque Mondiale \footnote{http://www.banquemondiale.org}, en 1960, l'espérance de vie moyenne était de 38,2 ans; en 2013, elle était de 63,4 ans. On a donc une progression moyenne de 25,2 sur la même période, soit plus du double de la France !

\begin{figure}[htbp]%insertion d'une image : définition d'un environnement figure
\begin{center}
\includegraphics[width=10cm]{images/EDV_France_Senegal.png} 
\end{center}
\caption{Comparaison de l'espérance de vie entre la France et le Sénégal de 1960 à 2013 - Source : La banque mondiale}
\label{Comparaison de l'espérance de vie entre la France et le Sénégal de 1960 à 2013 - Source : La banque mondiale} 
\end{figure}

On peut faire une autre comparaison avec un autre pays pauvre d'un autre continent : le Népal, en Asie. En 1960, d'après la banque mondiale, l'espérance de vie d'un népalais était de 38,5 ans; en 2013, elle était de 68,4 ans. on a donc une progression moyenne de 29,9 ans, encore plus grande qu'au Sénégal, sur la même période !

\begin{figure}[htbp]
\begin{center}
\includegraphics[width=10cm]{images/EDV_France_Nepal.png} 
\end{center}
\caption{Comparaison de l'espérance de vie entre la France et le Népal de 1960 à 2013 - Source : La banque mondiale}
\label{Comparaison de l'espérance de vie entre la France et le Népal de 1960 à 2013 - La banque mondiale}
\end{figure}

On constate donc que l'augmentation de l'espérance de vie est un phénomène mondial, globalisé mais très hétérogène d'un pays à l'autre.
Une question susbsiste cependant : comment expliquer une telle progression de l'espérance de vie et une telle hétérogénéité entre les pays ?

\subsubsection*{Element principal expliquant ce phénomène}

Au lendemain de la seconde guerre mondiale, le 24 octobre 1945, est créé l'Organisation des Nations Unies (ONU) \footnote{www.uno.org} : un de ses principal objectif est d'assurer le progrès social, l'application des droits de l'homme et la sécurité des citoyens du monde avec, a terme, l'ambition d'instaurer la paix mondiale. Pour mener à bien cette mission, elle est composée de plusieurs organismes dont celui de l'Organisation Mondiale de la Santé (OMS). 


La constitution de l'OMS a été votée le le 22 juillet 1946. L'organisation voit officiellement le jour le 7 avril 1948 (date anniversaire correspondant aujourd'hui à la Journée Mondiale de la Santé), son siège sociale étant situé à Genève, en Suisse.
Le rôle de l'OMS est de coordonner la santé internationale au sein du système des nations unis : elle établit, par exemple, des calendriers de recherche, fixe des normes et des critères dans le monde de la santé, articule des politiques sanitaires ou encore de surveiller la situation sanitaire et d'évaluer les tendances en matière de santé... De façon plus synthétique, l'OMS a pour vocation d'amener tous les peuples au niveau de santé le plus élevé possible. Or, la santé impact directement l'espérance de vie, plus qu'aucun autre facteur.


À titre d'exemple, de 1918 à 1919, après la première guerre mondiale, a eu lieu la pandémie espagnole. Selon l'Institut Pasteur, cette grippe aurait causé la mort de 50 millions de personnes. Des études plus récentes (Updating the accounts: global mortality of the 1918-1920 "Spanish" influenza pandemic.) ferait même état de 100 millions de morts. La première guerre mondiale a causé la mort de 9 millions de personnes et 20 millions ont été blessées : cela représente donc 29 millions de victimes. La grippe espagnole a donc été plus destructrice que la première guerre mondiale. En d'autre terme, l'espérance de vie pour un enfant né entre 1914 et 1918 était plus élevé que pour un enfant né entre 1918 et 1919. Cet exemple illustre bien l'impact des conditions de santé sur l'espérance de vie et ce, à une échelle mondiale !



Or, depuis sa création, fidèle à ses principes fondateurs, l'OMS s'est engagé à améliorer les conditions de santé de l'ensemble des citoyens vivant sur Terre. Par exemple, l'OMS a lancé de grandes campagnes de vaccination dans le monde qui, selon un de ses récents rapports, \og a été l'une des plus grandes révolution de la santé publique au XX\textsuperscript{ième} siècle \fg. Elle coordonne des études au niveau internationale, rassemble des base de données de différents pays et les exploite pour ensuite donner des axes de recherches précis, priorisés.  et pertinents aux différents pays. Un des exemples les plus illustratifs de cette démarche est la lutte contre l'obésité. Selon l'OMS, \og l'obésité chez l'enfant est associée à une gamme étendue de complications graves et à un risque accru de maladie prématurée.\fg Autrement dit, un enfant en situation d'obésité est un enfant avec avec une santé dégradé qui sera plus exposé à des problèmes tels que les maladies cardio-vasculaire qui pourraient influer de manière négative sur son espérance de vie. L'organisation a donc mis en place en 1986 une base de données mondiale sur la croissance et la malnutrition chez l'enfant. L'objectif était \og de compiler, de standardiser et de diffuser les résultats des enquêtes nutritionnelles dans le monde\fg. Cette base de données a ensuite été exploitée \og pour déterminer la prévalence mondiale et régionale et établir des estimations des nombres d?enfants présentant  un  retard  de  croissance,  un  déficit  pondéral,  une émaciation ou un excès pondéral\fg.

\begin{figure}[htbp]
\begin{center}
\includegraphics[scale=0.45]{images/ModeleFonctionnementOMS.png} 
\end{center}
\caption{Récapitulatif du fonctionnement de l'OMS pour la problématique de l'obésité}
\label{Récapitulatif du fonctionnement de l'OMS pour la problématique de l'obésité}
\end{figure}

En d'autre termes, grâce à l'OMS, les pays étaient en mesure de connaitre l'état d'obésité des enfants dans les pays et \og d'entreprendre [...] des efforts pour stopper toute progression supplémentaire de la prévalence de l'excès pondéral chez l?enfant\fg c'est-à-dire de mettre en place des politiques pour palier à ce problème.


Tous ces éléments nous mènent à la conclusion que c'est l'amélioration de l'accès aux soins qui a été le principal moteur de l'augmentation de l'espérance de vie. Cependant, les disparités sont très grandes et l'accès aux soins reste très inégal en fonction des populations.
Cela amène à s'interroger sur l'accès aux soins : quelle est la définition précise de ce concept ? quelles en sont les composantes qui expliqueraient de si grandes différences entre les populations dans le monde ?