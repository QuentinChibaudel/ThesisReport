%troisième chapitre de la partie théorique  : les dispositif médicaux et les aides techniques
\chapter{Les dispositifs médicaux et les aides techniques}

Cette partie est consacrée à l'étude des dispositifs médicaux d'un point de vue théorique et pratique ainsi qu'à l'étude des aides techniques.

\section{Les dispositifs médicaux}

\subsection{Définition}

La définition d'un dispositif médical a été donnée par une directive européenne votée en Juin 1993 : la directive 93/42/CE \cite{Directive93/42/CEE}. Elle définit un dispositif médical comme suit.

Est considéré comme dispositif médical : tout instrument, appareil, équipement, logiciel, matière ou autre article, utilisé seul ou en association, y compris le logiciel destiné par le fabricant à être utilisé spécifiquement à des fins diagnostique et/ou thérapeutique, et nécessaire au bon fonctionnement de celui-ci. Le dispositif médical est destiné par le fabricant à être utilisé chez l'homme à des fins de :
\begin{itemize}
\item diagnostic, prévention, contrôle, traitement ou d'atténuation d'une maladie,
\item diagnostic, contrôle, traitement, d'atténuation ou de compensation d'une blessure ou d'un handicap,
\item d'étude ou de remplacement ou modification de l'anatomie ou d'un processus physiologique,
\item maîtrise de la conception,
\end{itemize}
et dont l'action principale voulue dans ou sur le corps humain n'est pas obtenue par des moyens pharmacologiques ou immunologiques ni par métabolisme, mais dont la fonction peut être assistée par de tels moyens. (directive européenne 93/42/CEE) \cite{Directive93/42/CEE}.

\subsection{Catégorie de dispositifs médicaux}

Les dispositifs médicaux sont classés selon cinq catégories, une catégorie est associée à un type d'utilisation particulier et bien définit.
\begin{itemize}
\item les Dispositifs Médicaux Implantables Actifs (DMIA) : ils correspondent aux dispositifs implantés totalement ou partiellement dans le corps humain tel qu'un stimulateur cardiaque implantable par exemple,
\item les dispositifs Médicaux de Diagnostic In Vitro (DMDIV) : ils correspondent aux produits, réactifs, matériaux ou instruments destinés à réaliser des diagnostics in vitro, seuls ou en combinaison tel que certains diagnostique de trisomie par exemple
\item les Dispositifs Médicaux Fabriqués sur Mesure (DMFM) : ils correspondent aux dispositifs médicaux qui ne sont pas fabriqués en série mais selon une prescription écrite d'un praticien qualifié pour un patient désigné tel qu'une prothèse dentaire par exemple,
\item les dispositifs Médicaux destinés à la Compensation du Handicap : ils correspondent aux dispositifs médicaux qui agissent directement sur le handicap tel que les fauteuils roulants par exemple
\item enfin, tous les autres dispositifs médicaux : c'est une catégorie assez hétérogène contenant les autres dispositifs médicaux dont les caractéristiques ne permettent pas de les classer dans une des catégories précédentes. C'est par exemple le cas de certains dispositifs médicaux consommables comme les pansements.
\end{itemize}

\begin{figure}[htbp]%insertion d'une image : définition d'un environnement figure
\begin{center}
\includegraphics[scale=0.40]{images/CategoriesDM.png} 
\end{center}
\caption{Illustration de la catégorisation des dispositifs médicaux}
%\label{Illustration de la catégorisation des dispositifs médicaux}
\end{figure}

la figure est-elle utile ? pas sur ... --> voir avec Véronique


\subsection{Classification}

La directive 93/42/CE définit une classification des dispositifs médicaux en fonction de leur invasivité et des risques potentiels qu'ils présentent dans leur utilisation. Elle est bien distincte des catégories. En effet, plusieurs dispositifs appartenant à des catégories différentes peuvent appartenir à la même classe de dispositif médical.
Les dispositifs médicaux sont classés suivant quatre classes : la classe I (celle qui présente le moins de risque), la classe IIa, la classe IIb et la classe III (celle qui présente le plus de risque).
La figure suivante présente une illustration possible de cette classification et donne des exemples de dispositifs dans chaque classe. 

\begin{figure}[htbp]%insertion d'une image : définition d'un environnement figure
\begin{center}
\includegraphics[scale=0.85]{images/ClassesDM.jpg} 
\end{center}
\caption{Illustration de la classification des dispositifs médicaux}
\label{Illustration de la classification des dispositifs medicaux}
\end{figure}

Cette classification se base sur un ensemble de critères précis.

\subsection{Critères de classification}

Les critères de classification se décomposent en deux familles : les critères cliniques et les critères non cliniques.
\subsubsection{Les critères cliniques}
Les critères cliniques qui s'appliquent sont les suivant, par ordre d'importance décroissant (d'après la directive 93/42/CE) : 
\begin{enumerate}
\item caractère invasif : 
    \begin{itemize}
        \item s'il est implanté et destiné à rester dans le corps, la criticité est importante
        \item s'il est non implanté et pour un temps défini, la criticité est faible
    \end{itemize}
\item type d'invasivité : 
    \begin{itemize}
        \item si le dispositif pénètre le corps par un acte chirurgical, alors c'est un dispositif invasif de type chirurgical donc de criticité élevée
        \item si le dispositifs pénètre le corps par un orifice naturel alors c'est un dispositif invasif de type naturel donc de criticité moyenne
    \end{itemize}
\item la durée d'utilisation c'est-à-dire la quantification de la durée maximale durant laquelle le dispositif est susceptible d'être utilisé en continue. Il y a trois niveaus différents : 
    \begin{itemize}
        \item le dispositif de type long terme (c'est-à-dire une durée d'utilisation supérieure à un mois) est de criticité élevée
        \item le dispositif de type court terme (c'est-à-dire une durée d'utilisation comprise entre une heure et un mois) est de criticité moyenne
        \item le dispositif de type temporaire (c'est-à-dire une durée d'utilisation inférieure à une heure) est de criticité faible
    \end{itemize}
\item la possibilité de réutilisation 
    \begin{itemize}
        \item le dispositif est réutilisable
        \item le dispositif est à usage unique
    \end{itemize}
\item la visée du dispositif 
    \begin{itemize}
        \item le dispositif est à visée thérapeutique c'est-à-dire qu'il est utilisé pour soulager une blessure ou autre handicap
        \item le dispositif est à visée de diagnostique c'est-à-dire qu'il est utilisé pour établir un diagnostique. Ces dispositifs ne font pas l'objet d'une classification 
    \end{itemize}
\item la partie du corps en contact avec le dispositif
\end{enumerate}

Quand plusieurs règlent s’appliquent, la classe la plus élevée doit être retenue.
Ces règles établissent au total une cinquantaine de critères.

\subsubsection{Les critères non cliniques}

D'autres critères non-cliniques, non proposées par la directive 93/42/CE \cite{Directive93/42/CEE}, sont également à prendre en compte dans l'évaluation des dispositifs médicaux : 

\begin{itemize}
\item Un aspect management/logistique : impact sur l’organisation des soins, les pratiques professionnelles, l’accès au soin pour le patient
\item Un aspect psychologique/humain : respect de règles éthiques, condition de santé des professionnels, l’attente de la population/des patients, acceptation de la technologie
\item Aspect utilisabilité : ergonomie du dispositif, facilité d’apprentissage et d’utilisation, développement durable
\end{itemize}

Les critères (et notamment les critères cliniques) permettent donc de définir la classe d'un dispositif médical. Cependant, pour pouvoir être commercialisé au sein de l'union européenne (quelque soit le lieu de fabrication du dispositif), un dispositif doit être certifié grâce à l'obtention d'un marquage CE.
\subsection{Définitions des acteurs en relation avec les dispositifs médicaux}
Avant d'aborder les éléments permettant de comprendre le système d'apposition du marquage CE, il est important de bien comprendre qui sont les acteurs de ce monde et quels sont leurs rôles. L'ensemble des définitions et des rôles des différents acteurs est donné par la directive 93/42/CE.

\subsubsection{Fabricant}
Le premier rôle définit est celui de fabricant. La directive définit ce rôle comme suit : \og Personne physique ou morale responsable de la conception, de la fabrication, du conditionnement et de l'étiquetage d'un dispositif médical en vue de sa mise sur le marché en son nom propre, que ces opérations soient effectuées par cette personne ou pour son compte par une autre personne ; les obligations qui s'imposent au fabricant en vertu du présent titre s'imposent également à la personne physique ou morale qui assemble, conditionne, traite, remet à neuf ou étiquette des dispositifs médicaux, ou assigne à des produits préfabriqués la destination de dispositifs médicaux, en vue de les mettre sur le marché en son nom propre\fg .

Cette définition est relativement longue et complexe. Il y a deux points essentiels à retenir : 
\begin{enumerate}
\item s’il y a un changement de destination dans l’utilisation du dispositif médical alors on devient fabricant de dispositif médical
\item un dispositif médical doit être issu d’un seul et même fabricant. Si une entité assemble plusieurs dispositifs médicaux différents dont au moins un n’est pas issu de sa fabrication, alors il devient fabricant de dispositif médical
\end{enumerate}

Le fait d’être fabricant de dispositif médical implique une soumission à la législation, au mode de fonctionnement de l’organisation des soins en France, aux normes et aux modalités de conformités

\subsubsection{Destination}
La destination est définit par la directive comme suit : \og c'est l'utilisation à laquelle un dispositif médical est destiné d'après les indications fournies parle fabricant dans l'étiquetage, la notice d'instruction ou les matériels promotionnels\fg . 

Si la destination change, quel qu'en soit la raison, le dispositif médical est modifié et les critères doivent être réétudiés avant sa mise sur le marché.

\subsubsection{Mise sur le marché}
La mise sur le marché correspond à \og la première mise à disposition à titre onéreux ou gratuit d'un dispositif autre que celui destiné à des investigations cliniques, en vue de sa distribution  et/ou de son utilisation sur le marché communautaire qu'il s'agisse d'un dispositif neuf ou remis à neuf \fg .

La mise sur le marché correspond donc à fournir un accès  au dispositif par l'intermédiaire d'un distributeur.

\subsubsection{Distributeur}
Le rôle de distributeur est définit comme suit par la directive : \og toute personne physique ou moral se livrant au stockage de dispositifs médicaux et à leur redistribution ou à leur exportation, à l'exclusion de la vente au public \fg .

Le distributeur est donc l'entité qui met à disposition le dispositif une fois qu'il est mis en service.

\subsubsection{Mise en service}
La mise en service correspond à \og le stade auquel un dispositif est mis à disposition de l'utilisateur final, étant prêt à être utilisé pour la première fois sur le marché communautaire conformément à sa destination\fg . 

La mise en service correspond donc à la mise à disposition pour un utilisateur. L'utilisateur peut avoir des informations quant au fonctionnement du dispositif.

\subsubsection{Données cliniques et mandataire}
Les données cliniques sont \og les informations relatives à la sécurité et aux performances obtenues dans le cadre de l'utilisation clinique d'un dispositif\fg . Si les performances ne sont pas atteintes, il est possible de faire appel au mandataire.
Le mandataire correspond à \og toute personne physique ou moral établie dans la Communauté qui, après avoir été désigné par le fabricant, agit et peut être contactée par les autorités et les instances dans la Communauté en lieu et place du fabricant en ce qui concerne les obligations que la présente directive impose à ce dernier\fg . C'est le mandataire qu'il faut interpeller en cas de non correspondance entre les performances d'un dispositifs données et les performances réelles. 

\subsection{Marquage CE des dispositifs}
\subsubsection{L'obtention d'un marquage CE}
L'obtention et l'opposition du marquage CE se déroule en plusieurs.
La première étape consiste à définir la classe auquel appartient le dispositif en fonction des critères qui ont été cités précédemment.
La seconde étape consiste à que le dispositif est conforme aux normes fixés : 
\begin{itemize}
\item la norme EN ISO 13485 pour l'assurance qualité
\item la norme EN ISO 14971 pour l'analyse de risque
\item la norme EN ISO 14155 pour les investigations cliniques
\item la norme EN 550 pour la stérilisation
\item la norme EN ISO 10993 pour l'évaluation biologique
\item la norme EN 980 et GMND pour les symboles, la nomenclature et les informations à fournir par le fabricant
\end{itemize}

Il faut aussi prouver que le dispositif est conforme à la réglementation et à la législation en vigueur dans l'espace de commerce concerné.
Si l'ensemble de ces conditions est satisfait, le fabricant obtient une attestation de conformité et peut apposer le marquage CE sur son dispositif.

C'est le constructeur qui définit les performances à atteindre au moment de la création du DM. Le marquage est le plus souvent apposé par le fabriquant

\subsubsection{Importance du marquage CE}

Le marquage CE permet d'attester d'une évaluation du type bénéfice/risque supérieure à 1. Il constitue un passeport pour la libre circulation du dispositif au sein de l'Europe. C'est aux organismes de contrôle tel que l'ANSM\footnote{Agence Nationale de Sécurité du Médicament et des Produits de Santé} que les objectifs fixés par le constructeur sont atteints ou non. En cas d'échec, c'est à l'organisme de décider des sanctions à imposer au constructeur.

\subsubsection{Dispositifs étudiés dans le cadre de ce travail}

Il y a deux exceptions à l'ensemble des indications citées précédemment. Tout d'abord, les DMDIV\footnote{Dispositifs Médicaux de Diagnostique In Vitro} ne font pas l'objet de classification.

Les dispositifs médicaux de classe I sont auto certifiés par le fabriquant. Cela implique en particulier qu'il est maître de la mise sur le marché européen, ou non, de son dispositif médical. Dans le cadre des travaux présentés par la suite, seuls les dispositifs de classe I (fauteuils roulants, déambulateur, ...) seront étudiés.

Cependant, les dispositifs médicaux ne sont pas les seuls outils utilisés pour aider les personnes en situation de handicap. Il existe d'autres outils faisant face à une législation moins contraignante : les aides techniques.

\section{Les aides techniques}
\subsection{Définitions}
Le site handicap.fr définit comme \og tout instrument, équipement ou système technique adapté ou spécialement conçu pour compenser une limitation d'activité rencontrée par une personne du fait de son handicap, acquis ou loué par la personne pour son usage personnel \fg \footnote{accès à la définition : https://informations.handicap.fr/art-allocations-aides-51-1889.php}. D'après cette définition, les aides techniques diffèrent des dispositifs médicaux par le fait qu'elle est conçue spécialement pour une personne en situation de handicap alors que les dispositifs médicaux ont une utilisation plus large. Par exemple, un pansement, qui est un dispositif médical, peut être utilisé chez une personne n'étant pas en situation de handicap. De plus, l'aide technique est à destination d'un usage personnel là où un dispositif médical peut être à usage multiple. Des mêmes béquilles peuvent être utilisées par différentes personnes de manières successives.

Les aides techniques sont aussi définies par une norme ISO : la norme ISO 9999:2011 (dans la version la plus récente). D'après cette norme,\og est considéré comme un produit d’assistance (ou aide technique) tout produit (y compris tout dispositif, équipement, instrument et logiciel) fabriqué spécialement ou généralement sur le marché, destiné à :
\begin{itemize}
\item favoriser la participation
\item protéger, soutenir, entraîner, mesurer ou remplacer les fonctions organiques, les structures anatomiques et les activités, ou
\item prévenir les déficiences, les limitations d'activité et les restrictions de la participation
\end{itemize} 
utilisé par ou pour les personnes en situation de handicap\fg .
Cette définition marque aussi le fait qu'une aide technique est dédiée à la situation de handicap et à une personne en particulier. Elle complète l'autre définition par le fait qu'elle intègre un aspect social puisque l'aide technique est présente pour diminuer les restrictions et favoriser la participation à une activité. Mais elle rejoint la définition des dispositifs médicaux dans l'idée qu'elle peut compenser un handicap voire remplacer des fonctions organiques pour les plus poussées d'entre elles.

Il est donc possible de définir une aide technique comme un outil fabriqué sur mesure pour une personne en situation de handicap pour aider cette personne à compenser son handicap pour son usage personnel. Elle doit aider à favoriser son intégration dans la société en limitant la restriction et la participation à des activités. 


Il est difficile, de part leur diversité et leur spécificité, de classer les aides techniques de façon extrêmement précise et rigoureuse. Il existe tout de même différentes classifications des aides techniques.

\subsection{Classification et conditions d'utilisation des aides techniques}
Une classification existante des aides techniques est la classification proposée est l'arrêté 2-5 du code préfectoral
Cet arrêté classe les aides techniques en trois catégories :
\begin{itemize}
\item Le matériel figurant sur la LPPR \footnote{liste des produits et prestations remboursables}
\item Le matériel ne figurant pas sur cette liste
\item Les équipements d'utilisation courante ou comportant des éléments d'utilisation courante
\end{itemize}

Il définit également des conditions d'utilisations des aides techniques.
Pour en bénéficier la personne doit remplir les conditions relatives à la résidence, à l'âge et au handicap (conditions décrites dans le premier article de notre dossier). De plus, l'aide technique doit contribuer : 
\begin{itemize}
\item à maintenir ou améliorer l'autonomie de la personne pour une ou plusieurs activités ;
\item à assurer la sécurité de la personne handicapée ;
\item à mettre en œuvre les moyens nécessaires pour faciliter l'intervention l'aidant qui accompagne la personne handicapée
\end{itemize}

L'aide doit être suffisante et appropriée aux besoins et habitudes de la personne dans son environnement, ou de l'aidant lorsque l'aide facilite son intervention. Elle doit être utilisée régulièrement ou fréquemment. La personne doit être capable d'utiliser la plupart des fonctionnalités de l'aide technique.