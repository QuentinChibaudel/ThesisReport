%première partie de la thèse sur l'accès aux soins
\chapter{Accès aux soins}

Cette première partie est dédiée à la notion de l'accès aux soins. Avant d'étudier les enjeux liés à l'accès aux soins et de définir un cadre précis pour l'étude, une définition de l'accès aux soin sera étudiée. 


\subsection{L'accès aux soins}
\subsubsection{Définition du concept \og d'accès aux soins\fg}
Aucune définition du terme \og accès aux soins \fg{} n'existe dans le dictionnaire de la langue française. Par conséquence, on peut analyser le concept global d'accès aux soins selon deux axes d'approche : un premier axe orienté sur l'accès et un second orienté sur les soins.

\subsubsection{Le concept d'accès} 

D'après le dictionnaire de la langue française, le concept \og d'accès\fg{} renvoie à un moyen d'approche à un lieu, une ville ou encore une procédure. L'Oxford English Dictionnary \cite{oxforddictionnary} aborde le concept d'\og access \fg (\og accès\fg{}  en français) comme un droit d'entrée, d'admission et l'opportunité d'atteindre quelque chose.

Margaret Swoney \& Owen Barr (2004) \cite{SwoneyBarr2004} complètent ce concept d'accès en émettant l'idée que l'accès est associé à la possibilité d'entrer dans un système mais aussi de bénéficier des ressources disponibles.
Donabédian (1973) \cite{Donabedian1973} apporte d'autres informations supplémentaires par rapport à  ce concept : 
\begin{itemize}
\item l'aspect socio-organisationnel qui contient les attributs de ressources qui facilitent ou bloquent l'accès pour une personne
\item l'aspect géographique qui renvoie à la distribution spatiale et temporel d'accès à un élément
\end{itemize}
Selon lui, l'accès repose sur l'existence et la disponibilité de ressources à un instant t.

De manière complémentaire, pour Cerquiglini \& Olle (2008) \cite{Cerquiglini2015}, l'accès correspond au fait ou au droit d'atteindre une fonction, un état, une dignité désirée par une personne.

Par conséquent, l'accès fait référence à une volonté d'atteindre un lieu physique, géographique ou d'être inclut dans des démarches administratives, non matérielles. L'accès est aussi associé au droit de bénéficier des ressources existantes et disponibles, dans un lieu géographique précis, à un instant t pour permettre à une personne de réaliser ses souhaits. Des éléments de l'environnement peuvent interférer ou faciliter un accès à un ou des éléments, matériels ou non.

\subsubsection{Le concept de soins}
D'après le dictionnaire de la langue française, le concept de \og soins \fg{} renvoie à différents éléments. D'abord, cela correspond à la charge, au devoir de s'occuper de quelque chose. Il renvoie aussi à l'ensemble des actions destinées à entretenir son corps et à lui donner une belle apparence. Enfin, d'un point de vue purement médical, cela renvoie à un ensemble d'actions thérapeutiques.
D'après l'Oxford English Dictionnary \cite{oxforddictionnary}, le concept de \og care\fg{} (plus large que le concept de \og soin\fg{} en France mais qui est la notion qui s'en rapproche le plus), renvoie à la mise à disposition de ce qui est nécessaire à la bonne santé, au bien être et à la protection d'une personne ou d'un objet.
Cerquiglini \& Olle (2015) \cite{DictionnaireHachette} définissent le soin comme l'action, le moyen hygiénique ou thérapeutique ayant pour objectif l'entretien du corps, de la santé ou du rétablissement de celle-ci.

Le concept de \og soin\fg{} renvoie donc à une volonté de mettre à disposition un ensemble d'éléments (actions thérapeutiques, hygiéniques, listes d'actions, professionnels de santé, ...) pour maintenir, entretenir en bon état un objet physique et/ou une personne. L'objectif peut aussi être de rétablir l'état de l'objet ou de la personne prise en charge.


Il existe deux types de soins (Baggot, 1998, Lombrail P. \& Pascal J., 2005) \cite{Baggot1998} \cite{LombrailPascal2005} : 
\begin{itemize}
\item les soins primaires : (Baggot, 1998, Lombrail P. \& Pascal J., 2005) \cite{Baggot1998} \cite{LombrailPascal2005}
\begin{itemize}
\item selon Baggot (1998) \cite{Baggot1998}, ils correspondent aux besoins de la communauté à travers la maintenance des soins, la prévention, le diagnostique, le traitement et la rééducation
\item selon Lombrail \& Pascal (2005) \cite{LombrailPascal2005}, ils correspondent à l'entrée dans le système de soins
\item ils comportent donc l'éducation à la santé, la proportion de la santé et des services de prévention pour maintenir un haut niveau de santé et l'entrée dans le système
\end{itemize} 
\item les soins secondaires : (Lombrail P. \& Pascal J., 2005)\cite{LombrailPascal2005}
\begin{itemize}
\item ils comprennent tous les professionnels de santé spécialisés
\item il y a un besoin de développer une meilleure compréhension pour ces spécialistes
\end{itemize} 
\end{itemize}

\subsubsection{Le concept d'accès aux soins }

Une définition du concept de \og l'accès aux soins\fg{}  doit donc inclure les notions de volonté d'accès (lieu géographique, démarche administrative, respect de la dignité de la personne ou encore accomplissement personnel ) à un ensemble d'éléments dans le but de maintenir en bonne santé. Cette volonté peut être facilitée ou entravée par des éléments extérieurs dont la personne n'a pas forcément la maîtrise.
Une définition adéquate a été proposée par l'OMS  au 58\textsuperscript{ième} comité régionale de l'Europe qui s'est déroulé à Tbilissi, en Géorgie, du 15 au 18 septembre 2008 \cite{OMS2008}. Selon cette définition, l'accès aux soins peut être définit comme la facilité plus ou moins grande avec laquelle une population peut s'adresser aux services de santé dont elle a besoin. Les composantes économiques, physiques, culturelles, géographiques (entre autres) influent de manière positives ou négatives sur l?accès aux soins.

Selon Margaret Sowney et Owen Barr (2004) \cite{SwoneyBarr2004}, les éléments qui vont influer sur l'accès aux soins sont : 
\begin{itemize}
\item la distance (par exemple une personne qui vit dans un milieu rural, souvent associé à un désert médical, aura un accès aux soins plus difficile qu'une personne vivant en ville où il est plus facile de se rendre chez un professionnel de santé).
\item le temps (temps pour obtenir une consultation, temps de la consultation et temps passé avant d'obtenir une nouvelle consultation)
\item la compréhension (compréhension des symptômes, du traitement à suivre)
\end{itemize}

D'une manière plus générale, les éléments à prendre en compte pour permettre un accès aux soins de qualité est l'aménagement de l'environnement et la réflexion sur son accessibilité tant d'un point de vue urbain (accessibilité des bâtiments par exemple) que d'un point de vue matériel (moyens nécessaires à une bonne consultation disponibles) (Poppleweel \& al, 2014 \cite{Popplewell2014}; Graham \& Mann \cite{GrahamMann2008}, 2008; Mudrick \& al, 2012 \cite{Mudrick2012}; Merten \& al, 2015 \cite{Merten2015}; Carrillo \& al, 2011 \cite{Carrillo2011}).

Crouzatier (2010) \cite{Crouzatier2010} complète ces idées en supposant que l'accès aux soins suppose la réunion de deux conditions : 
\begin{itemize}
\item l'accès à la matérialité du soin c'est-à-dire les soins de base. Ils doivent être quantitatifs (structures, nombre de personnel soignant, médicament disponible) et qualitatif (personnel bien formé, accueil de qualité. Cet accès à la matérialité des soins doit se faire en assurant une protection des droits du patient (équité de traitement, respect de sa dignité en tant qu'être humain)
\item l'accès au financement des soins de santé. Cette aide peut être financière ou issue du système d'assurance maladie s'il existe
\end{itemize}

Il apparaît que le concept d'accès aux soins est fonction, de l'environnement urbain, de l'environnement matériel, du cadre légal dans lequel évoluent ces éléments et du fonctionnement du système de soins d'un pays. 

De plus, selon Aday \& Donabédian (2013 (ref chelou, demander à Véro si ok)) \cite{DayDonabedian2005} considèrent l'accès aux soins comme \og \textit{un déterminant de la question du système de santé. L'individu est doté d'une prédisposition à consommer des soins (ses préférences peuvent être culturellement déterminées) d'un besoin de consommer lié à son état de santé : il réalise alors sa demande de soins latentes plus ou moins biens selon les obstacles qu'il rencontre} \fg{}. L'accès aux soins concernent donc tous les individus d'une population à des degrés différents selon ses prédispositions et l'environnement matériel et humain qui l'entoure. Il convient donc de cadrer précisément cet environnement et d'en comprendre le fonctionnement.Dans notre cas, nous nous intéressons au cas particulier de la France.


\subsection{Fonctionnement du système de soins français}
\subsubsection{Cadre législatif en France}

Le cadre législatif français et son mode de fonctionnement sont très particuliers (notamment de part l'histoire riche et complexe de ce pays) et méritent une attention particulière.
Après la seconde guerre mondiale, les 4 et 19 octobre 1945, deux ordonnances sont promulguées par le nouveau gouvernement français, dirigé par le général Charles De Gaulle. Ces deux ordonnances créent l'organisation de la sécurité sociale. Elle est construite sur trois principes fondamentaux : l'égalité d'accès aux soins, la qualités des soins et la solidarité. Ces principes sont appuyés par le préambule de la constitution de 1946, qui pose les bases de la IV\textsuperscript{ième} république française : \og \textit{Il est institué une organisation de la Sécurité sociale destinée à garantir les travailleurs et leurs familles contre les risques de toute nature susceptibles de réduire ou de supprimer leurs capacités de gain, à couvrir les charges de maternité ou les charges de famille qu'ils supportent} \fg . 
En résumé, l'assurance maladie permet, en principe, à la plupart des personne une prise en charge au niveau de la santé tant dans le secteur privé que dans le secteur publique. Elle représente le principe d?égalité des citoyens (quelque soit le sexe, la religion, l'origine ethnique) face à loi et à la société.
En 1958, lors de l'instauration de la V\textsuperscript{ième} république française, le texte écrit en préambule réaffirme l'importance de la sécurité sociale : \og \textit{La Nation assure à l'individu et à la famille les conditions nécessaires à leur développement. Elle garantit à tous, notamment à l'enfant, à la mère et aux vieux travailleurs, la protection de la santé, la sécurité matérielle, le repos et les loisirs. Tout être humain qui, en raison de son âge, de son état physique ou mental, de la situation économique, se trouve dans l'incapacité de travailler a le droit d'obtenir de la collectivité des moyens convenables d'existence}\fg{} .

En 1967, face à des premières difficultés financières, une ordonnance divise la sécurité sociale en trois branches : maladie, vieillesse et famille. Chaque branche devient autonome, responsables de ses ressources et de ses dépenses.

Le 2 avril 1996, l'ordonnance numéro 96-344, promulguée par Alain Juppé, un nouveau mode de coordination est mis en place entre les différents acteurs : le gouvernement, le parlement, la sécurité sociale (chacune des branches) et les professionnels de santé. L'objectif est de mieux coordonner les soins entre ces acteurs afin de les dispenser de façon plus efficace et moins dépensière à la population. 
Toujours dans ce même objectif, une réforme de l'assurance maladie est votée le 4 Août 2004 : l'objectif est d'optimiser les dépenses de santé tout en préservant le caractère universel, obligatoire et solidaire du système. La figure suivante est une illustration de l'ensemble des acteurs et des organismes du système de soins français, la description rapide de leur rôle et des différents liens qui existent entre eux.

\begin{figure}[htbp]
\begin{center}
\includegraphics[scale=0.45]{images/FriseChronoSecuSociale.png} 
\caption{Historique de la sécurité sociale de 1945 à aujourd'hui} \label{Historique de la securite sociale de 1945 a aujourd'hui}
\end{center}
\end{figure}

En conclusion, la sécurité sociale, régit par des lois, a pour but d'assurer à chaque citoyen résidant sur le sol Français d'avoir un accès aux soins équitables et de qualité, sans discrimination quelconque par rapport à ses origines, sa sexualités, sa religion ou autres.
Ce système est relativement complexe et implique un grand nombre d'acteurs.

\subsubsection{Les principaux acteurs de la sécurité sociale}

Les acteurs du système présenté sur la figure \ref{GalaxieDM} se divise en deux catégories principales : les organismes gérant l'aspect financier d'une part et les organismes gérant l'aspect régulation d'une autre part. 

\begin{figure}[htbp]
\begin{center}
\includegraphics[width=15cm]{images/GalaxieActeurSecuFrance.png}
\end{center}
\caption{\label{GalaxieDM} La galaxie française des dispositifs médicaux}
\end{figure}

Tous les acteurs présentés dans la figure \ref{GalaxieDM} se coordonnent autour de deux éléments principaux de fonctionnement : l'hôpital et la ville.


\subsubsection{Le fonctionnement de la ville et de l'hôpital}

En fonction du domaine d'action, différents acteurs interviennent. 
Le gouvernement, par l'intermédiaire du ministère de la santé, définit le budget alloué à la sécurité sociale et les principaux axes d'actions à mener.
Ensuite, c'est la Haute Autorité de Santé qui s'assure de la bonne mise en oeuvre de cette politique. Sous la tutelle de la HAS, le Centre National d'Évaluation des Dispositifs Médicaux et des Technologies de Santé (CNEDiMTS) émet des avis sur les éléments à rembourser ou non. Cet avis doit être cohérent et en adéquation avec le budget et la politique mise en place par le gouvernement. Ces organismes agissent dans le domaine de l'hôpital et dans le domaine de la ville.

Du côté de la ville, le Comité Économique des Produits de Santé (CEPS), fait une proposition sur le prix et les remboursement des médicaments et des dispositifs médicaux à la CNEDiMTS. Celle-ci prend alors ces éléments en compte pour faire des propositions à la HAS.
Au niveau régional, les Agences Régionales de Santé (ARS) s'assurent de la bonne mise en oeuvre de la politique définit, tant au niveau de l'hôpital que de la ville.
Du côté de l'hôpital, elle cherche à optimiser au maximum le budget qui lui est alloué. Elle travaillent essentiellement avec des professionnels de santé salariés spécialisés dans un domaine précis. Dans le domaine de l'hôpital, les cadres de santé ont à faire avec des prestataires de services référencés.
Du côté de la ville, l'ARS possède un droit de regard et de surveillance sur la Caisse d'Assurance Retraite et de la Santé au Travail (CARSAT) de la même région. La CARSAT assure le remboursement et les cotisations à la retraite des salariés. Les CARSAT sont appuyés dans leur tâche par la Caisse Primaire d'Assurance Maladie (CPAM). La grande difficulté est que le budget est difficile à prévoir : en effet, c'est la CARSAT qui assure le remboursement des consultations des médecins libéraux. Or, il est complexe de prévoir le nombre de consultations, celles-ci pouvant énormément varier en fonction des années, de la maitrise des épidémies, ...

\begin{figure}[htbp]
\begin{center}
\includegraphics[scale=0.75]{images/ActeursSecuSociale.png} 
\end{center}
\caption{\label{ActeursSecuriteSociale}Principaux acteurs dans le système de la sécurité sociale français en fonction des domaines}
\end{figure}

La figure \ref{ActeursSecuriteSociale} a été établit suite à des entretiens menés avec un expert du système de soin français. Elle a été validée par ce dernier.

Les acteurs diffèrent en fonction de la ville et de l'hôpital. Il en va de même pour le mode de fonctionnement de chaque  domaine.

Au niveau de l'hôpital, le système est très rigoureusement définis : 
\begin{itemize}
\item à chaque pathologie est associé un ou des professionnels de santé,
\item  la procédure de diagnostique, de suivi sont très codifiées
\item on y applique la Tarification À l'Acte (T2A) : pour chaque action, un coût précis est associé
\item le matériel (notamment dispositif médical) appartient à l'hôpital
\end{itemize}

Du fait de la plus grande difficulté de prévision du système de la ville évoqué précédemment, le système est assez différent de celui de l'hôpital : 
\begin{itemize}
\item la personne consulte un médecin généraliste, pas forcément spécialisé dans un domaine extrêmement précis
\item le médecin fait au mieux pour établir un diagnostique, il possède une liberté assez grande
\item le prix d'une consultation est toujours le même
\item le matériel est remboursé (en totalité ou partiellement) au patient et ce dernier possède alors le matériel
\end{itemize}
La figure \ref{DifferenceVilleHopital} présente un résumé du fonctionnement de ce système et des différences entre la ville et l'hôpital. 


\begin{figure}[htbp]
\begin{center}
\includegraphics[scale=0.75]{images/VilleEtHopital.png} 
\end{center}
\caption{\label{DifferenceVilleHopital} Différence de fonctionnement entre le système de la ville et de le système de l'hôpital}

\end{figure}

La figure \ref{ActeursSecuriteSociale} a été établit suite à des entretiens menés avec un expert du système de soin français. Elle a été validée par ce dernier.

En conséquence, pour le cas spécifique de la France, on peut ajouter à la définition précédente de l'accès aux soins comme la facilité de communication et d'accès à l'information avec les différents organismes et la facilité d'accès aux différents systèmes.

Malgré cette complexité de fonctionnement, le système français repose sur des valeurs de solidarités, d'égalité d'accès aux soins (c'est-à-dire un accès gratuit pour tous et n'importe où en France) et de qualité des soins (autrement dit, pouvoir accéder à la ville ou à l'hôpital de manière aisée en fonction de sa situation). Cependant, de grandes inégalités de soins persistent dans notre pays.


\subsubsection{Composantes de l'accès aux soins}

Dire qu'il y a beaucoup de composantes dans l'accès aux soins. Essayer de faire une petite carte conceptuelle peut être pas mal non? 
--> est-ce le bon endroit pour mettre cet élément ? pas sur, à réfléchir

\subsection{Difficultés d'accès aux soins}
Pour comprendre les inégalités d'accès aux soins en France ou ailleurs, et les enjeux liés à ce phénomène, il est important de s'interroger sur les concepts d'équité, d'égalité, d'iniquité et d'inégalités.

La HAS\footnote{Haute Autorité de Santé : autorité publique indépendante qui contribue à la régulation du système de santé } (2009) \cite{HAS2009} classe les difficultés d'accès aux soins en deux catégories : 
\begin{enumerate}
\item les obstacles généraux liés à la personne, au défaut de compétence ou de disponibilité des professionnels, l'inaccessibilité / l'inadaptation des services disponibles en fonctions des besoins, des difficultés rencontrées par les aidants familiaux ou des problèmes de solvabilité dans l'accès aux soins
\item les obstacles particuliers selon si la personne vit en établissement ou à son domicile, selon le type de handicap (psychique, mental, sensoriel), les types de soins exigés, l'avancée en âge ou encore l'éventuel situation de précarité
\end{enumerate}

Ce constat de la HAS montre que l'accès aux soins est différent selon chaque individu. Cela implique que chaque personne a des besoins différents et que les soins apportés doivent être spécifique et adaptés. De plus, comme le stipule la Déclaration Universelle des Droits de l'Homme adoptée le 10 Décembre 1948 \cite{DUDH} \og \textit{Toute personne a droit à un niveau de vie suffisant pour assurer sa santé [...] notamment pour les soins médicaux ainsi que pour les services sociaux nécessaires} \fg{}


Cela signifie que malgré la diversité des individus, chacun a droit à avoir un accès aux soins égale et équitable. Cependant, ces termes restent à définir.

\subsection{L'égalité et l'équité}
\subsubsection{L'égalité}
Dans son sens le plus stricte, l'égalité représente une équivalence entre deux termes ou plus, évaluée sur une échelle de valeur, mesurée avec un degré de similitude d'identité des termes, ou sur des critères de préférences. (Hutmacher \& al, 2001) \cite{Hutmacher2001}. Dans cette définition, l'égalité repose sur la comparaison d'au moins deux entités similaires.
D'après le dictionnaire Larousse \cite{DictionnaireLarousse} de la langue française, le concept d'égalité fait référence à une absence de discrimination entre les êtres humains, sur le plan de leurs droits. Friant (2012) \cite{Friant2012} va dans le sens de cette définition ajoutant que l'égalité concerne des avantages ou des désavantages objectifs, mesurables. 
La combinaison de ces deux définitions implique que les entités à comparer sont des êtres humains. Cela implique également que la mesure doit se baser sur une différence de caractéristiques entre ces deux entités : plus cette différence est proche de zéro, plus les individus sont égaux. Il reste tout de même compliqué de définir des critères objectifs de comparaison permettant d'établir une réelle égalité entre deux entités. Si une personne est non-voyante et qu'elle rencontre une personne n'ayant aucun trouble visuel particulier. Ces deux personnes peuvent échanger oralement et se comprendre. Le critère à mesurer est le suivant : ces personnes sont-elles capables de recevoir l'information de l'autre ? Le résultat est soit oui, soit non. Dans cet exemple, le résultat est oui donc il n'y a pas de différences entre ces deux personnes par rapport à ce critère. Il n'y a alors, \textit{à priori}, aucune différence entre elles puisqu'elles se comprennent parfaitement. Elles sont égales au sens de la définition proposée précédemment. Ensuite, une tierce personne a enregistré ce débat et en établit une retranscription manuscrite. La personne non-voyante se retrouve alors discriminé car elle n'a pas accès à l'information. Une inégalité se créé entre ces deux personnes. Par conséquent, le fait de mesurer une différence et d'essayer de la réduire au minimum n'est pas suffisant pour assurer une égalité entre les personnes. Vedel (1990) \cite{Vedel1990} exprimait ses doutes quand à cette méthode estimant qu'\og au fond, jamais personne n'en a[vait] donné une définition assurée \fg{} puisqu'elle était \og d'abord une intuition exigeante, aux limites sans cesse repoussées qui [...] ne peut être mise en formule rationnelle\fg{}. Il devient alors compréhensible que définir l'égalité comme une différence nulle entre des caractéristiques de deux entités similaires est incomplet et manque de rigueur. 

Le professeur Rivero \cite{Rivero1966} décline le concept d'égalité sous différentes formes : 
\begin{enumerate}
\item l'égalité \textbf{devant} la règle de droit : concept très abstrait ne prenant pas en compte la situation de la personne
\item l'égalité \textbf{dans} la règle de droit : concept qui tient compte des situations des lesquelles se trouvent les personnes concernées
\item l'égalité \textbf{par} la règle de droit : concept qui cherche à réduire les inégalités. Elle se place dans un esprit de redistribution et de correction. Il y a recherche de l'égalité par différence de traitement.
\end{enumerate}

L'égalité devant la règle de droit ne sera pas considérée dans ce cas d'étude. En effet, l'exemple précédent a montré que l'égalité est fonction de la situation. Ne pas la prendre en compte n'est pas cohérent. En revanche, l'étude du concepts d'égalité dans la règle de droit car elle tient compte de la situation des personnes. L'étude du concept par la règle de droit est tout aussi intéressante car elle renverse le point de vue : l'égalité consiste à combler les inégalités entre deux personnes par une différence de traitement. Ce point de vue rejoint celui de Plottu (2010) \cite{Plottu2010} qui se demandait si la question de l'égalité ne devait être \og qu'un objectif désirable \fg{} et si \og tous les ménages [devaient] être aidés de façon égales ?\fg{}. 

En reprenant l'exemple précédent, inégalité apparaît avec le changement de situation : c'est le fait de passer d'un mode d'échange oral à un mode d'échange écrit qui créé l'inégalité. Cela justifie également le renversement de point de vue l'égalité étant un objectif à atteindre par différence de traitement. En fournissant un outil adapté à la personne non voyante, elle sera remise sur un pied d'égalité avec son interlocuteur. Le traitement est inégal car seule la personne non voyante reçoit une aide spécifique. En revanche, cette différence de traitement permet de ramener les protagonistes à égalité par rapport à la situation. 

En définitive, l'égalité est un état de non différence par rapport à une situation et des critères objectifs et mesurables entre au moins deux protagonistes. Cette absence de différence peut être atteinte par une différence de traitement. Il devient alors de s'interroger sur la nouvelle variable introduite dans cette définition : la différence de traitement entre deux protagonistes. Comment la mettre en place ? Comment mesurer son impact ? Comment l'adapter aux différentes situations envisageables ?


\subsubsection{L'équité}
Pour répondre à ces questions, il convient d'étudier le concept d'équité. En effet, comme le confirme un rapport publié par Alain Mine \cite{MineRapport1994} en 1994, le mot \og équité\fg{} est devenu \og le nouveau maître mot (du langage politique et social) comme l'égalité l'a été pour le modèle après guerre\fg{}. Une année plus tard, en 1995, le rapport remis par J.B.De Foucauld \cite{Foucauld1995} que \og La France a vécu jusqu'ici sur un modèle égalitaire simple : accroître de façon uniforme les droits juridiques ou sociaux, réduire les inégalités de revenus, développer pour tous les prestations sociales. [...] La commission souhaite qu'il soit fondé sur le principe d'équité par opposition à l'aspiration égalitaire qui a bercé toute l'histoire sociale de l'après-guerre\fg{}. Ces rapports confirment le besoin de changer de point de vue par rapport à l'égalité en passant par le concept d'équité. D'après le dictionnaire Larousse de la langue française \cite{DictionnaireLarousse}, l'équité est \og une qualité consistant à attribuer à chacun ce qui lui est dû par référence aux principes de la justice naturelle\fg{}. Le dictionnaire de l'académie française \cite{DictionnaireAcademieFrancaise}, quant à lui, définit l'équité comme \og un sentiment naturel, spontané du juste et de l'injuste\fg{}. Le dictionnaire Robert \cite{DictionnaireRobert} affine cette définition en précisant que l'équité est une notion de justice naturelle dans l'appréciation de ce qui est dû à chacun. C'est aussi la conception d'une justice naturelle qui n'est pas inspirée par les règles du droit en vigueur. L'équité permet donc de rétablir un état de justice entre plusieurs personnes sans que les moyens mis en \oe{}uvre ne soit définit par des règles précises. Cette définition reste relativement vague et ne permet pas de travailler sur l'utilisation du concept d'équité pour atteindre un état d'égalité. 


Dans son rapport, J.B. Foucauld \cite{Foucauld1995} affine son approche de l'équité. Selon lui, l'équité peut être étudiée sous deux angles différents : 
\begin{enumerate}
\item l'équité renvoie à justice commutative et répond à la maxime \og à chacun selon dû\fg{}. Elle implique que chacun reçoive l'équivalent de son apport et participe au fonctionnement des charges sociales en fonction du coût qu'il faut supporter à la société
\item l'équité peut aussi renvoyer à l'idée de justice distributive et répond à la maxime \og à chacun selon ses besoins\fg{}. Basée sur le principe de redistribution en fonction de favorisée / défavorisée en fonction des situations de chacun. Elle recherche l'égalité formelle et non pas réelle. Chacun doit participer au financement des charges sociales en fonction de ses capacités
\end{enumerate}

Ces deux approches ont un élément en commun : l'équité est adapté en fonction de chaque personne, au cas par cas. Cela confirme que l'équité ne suit pas des lois précises. La première approche implique une participation adaptée à la situation de chaque individu : la contribution des individus est proportionnelle à leur capacité. La deuxième approche cherche à compenser des différences entre les individus en répartissant les ressources entre les individus en fonction de leur besoins respectifs. L'objectif de la deuxième approche est d'établir un état d'égalité entre les différents individus concernés.

Ainsi, comme le souligne Plottu (2010) \cite{Plottu2010}, \og Le recours à un principe d'équité [...] passe désormais par le principe de discrimination positive : certaines catégories doivent être plus particulièrement aidées\fg{}. L'accès à l'égalité passe donc par la réflexion de la mise en place d'une solution équitable. Par conséquent, \og l'égalité et l'équité sont ainsi deux concepts bien distincts bien qu'intimement associés.\fg{} (Hutmacher \& al, 2001) \cite{Hutmacher2001}. Il s'agit maintenant de comprendre comment sont liés ces deux concepts et comment ils interagissent.

\subsubsection{Alors, comment lié l'équité et l'égalité ? }

Comme l'indique Friant (2012), la notion d'équité est liée à des jugements sur le caractère juste d'une situation alors que la notion d'égalité ne l'est pas. Ainsi, une solution égale est une solution qui ne prend pas en compte la spécificité d'un individu (sa localisation géographique ou sa position sociale par exemple). Une solution égale est proposée à tout le monde, sans distinction. L'équité prend notamment en compte la situation de la personne pour adapter l'aide à apporter.

Par conséquent, une solution égale peut être source d'inégalité. Et, de manière paradoxale, une solution équitable est une solution inégale qui permet de ramener les gens sur un plan égalitaire comme l'illustre la figure \ref{EgaliteVsEquite}.

\begin{figure}[htbp]
\begin{center}
\includegraphics[width=10 cm]{images/egalite_equite.jpg} 
\end{center}
\caption{Illustration des concepts d'égalités et d'équité (http://www.hierophanie.net/wp-content/uploads/2016/02/egalite-equite.jpg)}
\label{EgaliteVsEquite}
\end{figure}

Cette exemple illustre très bien la différence et la complémentarité de ces concepts d'égalité et d'équité. 

Dans la solution dite \og égalité \fg{}, il n'a pas été tenu compte de la différence de taille des protagonistes. Par rapport au modèle de Rivero explicité ci-dessus, ce cas de figure correspond à une égalité devant la règle de droit : il n'est pas tenu compte de la situation. De plus, parmi les propositions de Foucauld, le concept d'équité considéré est celui qui renvoie à une justice commutative où chacun est aidé selon son dû. Ici, le dû est identique pour chaque personne. Ainsi, la solution n'aide pas la personne qui en avait le plus besoin (ici, la personne de plus petite taille) et aggrave les différences entre elles. Cette solution égale et équitable est donc source d'inégalités.  


Au contraire, dans la solution dite \og équité \fg{}, la solution est adaptée en fonction de la taille des différents protagonistes. Par rapport au modèle de Rivero, dans cette situation, deux formes d'égalité sont considérées : l'égalité dans la règle de droit (qui tient compte de la situation de chacun) et l'égalité par la règle de droit (qui cherche à réduire les inégalités pour atteindre l'égalité). Par rapport aux propositions de Foucauld, le concept d'équité considéré est celui qui renvoie à l'idée de justice distributive où chacun est aidé selon ses besoins. En conséquence, cela permet à chaque protagoniste de l'image d'être placé sur le même plan, d'être mit à égalité.

Pour garantir cette idée d'égalité des citoyens, promu par la sécurité sociale, il faut donc proposer des solutions équitables c'est-à-dire des solutions adaptées aux besoins de chacun et non pas une seule et même solution pour tous. Cependant, mettre en place une solution équitable n'est pas toujours évident et il peut en découler des inégalités.

Le lien entre les concepts d'équité et d'égalité est expliqué. Il faut désormais l'appliqué au champ étudié : l'accès aux soins.

\subsubsection{Équité, égalité et accès aux soins}

Le concept d'équité de soins est assez difficile à définir (Evandrou et al, 1990) \cite{Evandrou1990}. Selon eux, ce terme décrit une égalité d'accès, d'utilisation et de résultats. Il est également associé à l'accès égal aux différents services et les soins donc les personnes bénéficient. Selon eux, l'équité est donc associée à un réel gain lors de l'entrée de la personne dans e système mais aussi sur les conséquences du traitement suivi.
\og L'accès équitable\fg{} peut être vu comme \og l'égalité d'opportunité\fg{} (Nutley \& Osborne, 1994) \cite{NutleyOsborne1994}. Cela correspond à un processus de suppression de barrières, de préjugés, de politiques discriminatoire avec pour objectif d'atteindre un accès équitable. L'idée est la recherche d'égalité par suppression des inégalités en proposant des solutions équitables. 
L'équité d'accès aux soins est donc une solution qui s'adapte à la personne, à l'environnement dans lequel elle évolue afin de supprimer les obstacles (préjugés, situation sociale, origine, ...) auxquels elle peut faire face. L'objectif est de lui permettre un réel gain dans l'amélioration de son état de santé et de son suivi. Chaque solution proposée est donc propre à une personne et ce, afin de répondre à la volonté de la politique de soins française, à savoir d'assurer l'égalité et l'accès aux soins pour tous.
Si cette équité n'est pas assurée, des inégalités peuvent apparaître.

\subsubsection{Différents types d'inégalités}

L'inégalité d'accès aux soins correspond à des besoins de soins non suivi d'accès ou lorsque les soins délivrés ne conduisent pas à des résultats de santé identique. 
De plus, selon Jonathan Mann (1998) \cite{Mann1998}, les différents types d'inégalités d'accès aux soins sont issus du même déterminisme : la \og socioparésie\fg c'est-à-dire l'incapacité d'appréhender la dimension sociale des problèmes de santé.

Ces inégalités se déclinent sous deux-formes : 
\begin{itemize}
\item inégalités par \og omission\fg : inégalités produites par le fonctionnement en routine du système de soins tel qu'il est (notamment le fait qu'il soit orienté sur du soin curatif et très balkanisé). Elles ne sont pas intentionnelles et sont le résultat soit de la non-reconnaissance de soins soit le la non satisfaction au moment de son identification.
\item  inégalités par \og construction\fg : elles sont le résultat de l'absence de prise en compte des inégalités dans l'élaboration de programme de soins ou de recommandations de pratiques médicales. Ces inégalités sont d'autant plus dommageables qu'un minimum de conscience de la socioparésie ambiante permettrait de les éviter à la source
\end{itemize}

En conséquence, il apparaît que la dimension sociale et la prise en compte des caractéristiques de l'individu sont des éléments essentiels à prendre en compte dans l'accès aux soins et dans la proposition de solutions équitables.

Des populations sociales spécifiques seraient plus touchées par ces non considérations et ces non adaptations de l'environnement. Les personnes en situation de handicap seraient particulièrement touchées par ce manque d'accessibilité (difficultés de communication de coopération dans les soins, dans l'expression du besoin) (Buissière, 2016) \cite {Bussiere2016}.

\subsection{Une difficulté d'accès aux soins plus accrues pour les personnes en situation de handicap mental}

Les personnes en situation de handicap mental ont un besoin d'accès aux soins plus élevés que le reste de la population (DREES, 2011 \cite{DREES2011}; OMS, 2011 \cite{OMS2011}; Lengagne \& al, 2015 \cite{Lengagne2015}). Ces personnes sont en moins bonne santé que le reste de la population : le taux de morbidité et de mortalité y est plus élevé que dans le reste de la population (Prince \& al, 2007) \cite{Prince2007}. En effet, cette population fait face aux mêmes problèmes de santé que le reste de la population. Mais elles sont plus vulnérables à certaines maladies chroniques (hypertension, maladie cardio-vasculaire, ...) en raison d'un comportement à risque plus élevé comme une activité physique plus faible, une consommation d'alcool et de tabac plus élevé ou une mauvaise alimentation (Havercamp, 2004; Rimmer, 2008). En résumé, cette population rencontre de grandes difficultés dans un accès aux soins de qualité (DRESS, 2011 \cite{DREES2011}; OMS, 2011 \cite{OMS2011}; Lengagne \& al, 2015 \cite{Lengagne2015}) et est marginalisée dans cet accès (Jacob \& al, 2013 \cite{Jacob2013}).

L'accès aux soins de types préventifs est particulièrement illustratif de cette situation. Les personnes en situation de handicap reçoivent moins de préventifs, notamment au niveau ophtalmologique que le reste de la population (Kancherla \& al, 2013 \cite{Kancherla2013}; Lengagne \& al, 2014 \cite{Lengagne2014}). 

De manière globale, les personnes en situation de handicap ont un état de santé bucco-dentaire particulièrement dégradé impliquant d'autres pathologies comme les caries ou des mauvaises déglutitions. Par conséquent, cette population fait face à une accumulation d'éléments qui rend la situation très complexe (Jensen \& al, 2008 \cite{Jensen2008}; Lengagne \& al, 2015 \cite{Lengagne2015}; Pezzementi \& al, 2005 \cite{Pezzementi2005})

Ces difficultés sont aussi propres à chaque personne en fonction de son sexe. Par exemple, les femmes en situation de handicap ont moins de chance de bénéficier d'un dépistage du cancer du sein et du col de l'utérus(Andresen \& al, 2013 \cite{Andresen2013}). Les hommes, quant à eux, ont moins de chance de bénéficier d'un dépistage du cancer de la prostate (Hoffman \& al, 2007 \cite{Hoffman2007}; Ramirez, 2005 \cite{Ramirez2005})

L'enquête de l'OMS (2004) \cite{OMS2004} révèle qu'il y a significativement plus de personnes demandeuses de soins de santé ne recevant pas les soins nécessaires parmi les personnes en situation de handicap que parmi les personnes ne présentant pas de situation de handicap.

Comme le résume Beatty \& al (2003) \cite{Beatty2003}, les personnes en situation de handicap sont souvent en situation de précarité socio-économique. Cela est du en partie, comme le précise Trani \& al (2012) \cite{Trani2012}, au fait qu'il existe des liens de causalité à double sens entre le handicap, la vulnérabilité socio-économique et le mauvais état de santé. Chaque élément provoque et accentue l'importance d'un autre. L'exemple le plus représentatif de ce cercle vicieux sont les maladies chroniques : elles ont un comportement plus à risque entraînant une plus grande vulnérabilité à ces maladies. Cela est accentué par l'état de précarité socio-économique. Or cet état de précarité accentue la vulnérabilité aux maladies chroniques.Donc, en plus d'être une population ayant un besoin de soin supérieur au reste de la population, de présenter plus de risque, cette population est socialement défavorisée par rapport à l'accès aux soins et au système de santé, quel qu'il soit. De plus, comme le confirme Buissière (2016) \cite{Bussiere2016}, \og \textit{le sujet de recours aux soins en général des personnes en situation de handicap reste sous-exploré}\fg{}. Il en résulte que le profil de ces personnes n'est pas connus des professionnels de santé. Cette non-connaissance de cette population est un frein supplémentaire dans l'accès aux soins (Popplewell \& al, 2014 \cite{Popplewell2014}; Graham \& al, 2008 \cite{GrahamMann2008}; Merten \& al, 2015 \cite{Merten2015}; Carrillo, 2011 \cite{Carrillo2011}).

Enfin, les personnes en situation de handicap mental sont particulièrement exposées à des risques de violences morales ou sexuelles renforçant un isolement sur eux-mêmes et ajoutant une difficulté supplémentaire dans l'accès aux soins (Peckham, 2007, \cite{Peckham2007}).

La population des personnes en situation de handicap mental mérite une attention particulière par rapport à cela. Il convient donc de définir précisément les caractéristiques de cette population et de comprendre quelles particularités elle présente pour comprendre le problème et proposer des solutions pertinentes et adéquates.

