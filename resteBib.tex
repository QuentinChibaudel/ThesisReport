

@article {handicap,
 TITLE = {Equity of access to health care for people with learning disabilities},
 AUTHOR = {Swoney M. AND Barr O.},
 YEAR = {2004}, 
 JOURNAL = {Journal for learning disabilities}, 
 PAGES = {247-265}}


Organisation Mondiale de la santé (OMS). (1988). Classification internationale des handicaps: déficiences, incapacités et désavantages: un manuel de classification des conséquences des maladies. Paris: CTNERHI-INSERM.


Donabedian,A. \textit{Aspects of Medical Care Administration Cambridge: Harvard University} Press,1973

Pierre Lombrail, Jean Pascal, « Inégalités sociales de santé et accès aux soins », Les Tribunes de la santé 2005/3 (n 8), p. 31

BAGGOT, R. (1998) Health and Health Care in Britain, 2nd edn. Basingstoke: Macmillan

OMS, comité régional de l’Europe, cinquante-huitième session, Tbilissi (Géorgie), 15-18 septembre 2008-39

 Mann J., « Santé publique : éthique et  droits de la personne », Santé publique, 10, 239- 250, 1998
 .
 Borioli, J., & Laub, R. (2007). Handicap : de la différence à la singularité. Chêne-Bourg,Suisse.

Hamonet, C. (2010). Les personnes en situation de handicap. Presses universitaires de France.

Diderot 1749 - lettre sur les aveugles à l'usage de ce ceux qui y voit

Kant - 1749 - Critique de la raison pure

Kaspi A. Franklin D. Roosevelt: le Grand livre du mois; 2009. 647 p

CERQUIGLINI B. et OLLE J., Dictionnaire universel, 5e édition, HACHETTE Edicef, Vanves, 2008, p. 8 et 1165

ADAY A. et DONABEDIAN A., Conception et définition de l'accès aux soins de santé [en ligne]. Rwanda : 2005 [ref. du 01/07/2013]. Disponible sur : http://www.memoireonline.com/07/08/1302/m_contribution-mutuelles-sante-accessibilite- population-services-de-sante3.html 

EVANDROU, M., FALKINHAM, J., LEGRAND, J. & WINTER, D. (1990 Equity in Health and Social Care. London: London School of Economics.

NUTLEY, S. & OSBORNE, S. P. (1994 The Public Sector Management Handbook. Harlow: Longman.

CROUZATIER J., « L’accès aux soins des migrants au regard du droit internationale », L’accès aux soins, Acte du colloque n°8, Presse de l’Université de Toulouse 1 Capitole, Toulouse, 2010, p 147


POPPLEWELL Nicola T A, RECHEL Boika P D et ABEL Gary A, 2014, « How do adults with physical disability experience primary care? A nationwide cross-sectional survey of access among patients in England », BMJ Open, 8 août 2014, vol. 4, no 8.

UNITED NATIONS, 2007, World Population Prospects The 2006 Revision, New York, NY, UN, Department
of Economic and Social Affairs.

GRAHAM Catherine Leigh et MANN Joshua R., 2008, « Accessibility of primary care physician practice sites in South Carolina for people with disabilities », Disability and Health Journal, octobre 2008, vol. 1, no 4, p. 209-214.

MUDRICK Nancy R., BRESLIN Mary Lou, LIANG Mengke et YEE Silvia, 2012, « Physical accessibility in primary health care settings: Results from California on-site reviews », Disability and Health Journal, juillet 2012, vol. 5, no 3, p. 159-167.

MERTEN Julie Williams, POMERANZ Jamie L., KING Jessica L., MOORHOUSE Michael et WYNN Richmond D., 2015, « Barriers to cancer screening for people with disabilities: A literature review », Disability and Health Journal, janvier 2015, vol. 8, no 1, p. 9-16.

CARRILLO J. Emilio, CARRILLO Victor A., PEREZ Hector R., SALAS-LOPEZ Debbie, NATALE-PEREIRA Ana et BYRON Alex T., 2011, « Defining and targeting health care access barriers », Journal of Health Care for the Poor and Underserved, mai 2011, vol. 22, no 2, p. 562-575.

HAS, 2009, Audition publique: Accès aux soins des personnes en situation de handicap (22 et 23 octobre 2008), Paris, France, Haute Autorité de Santé

Déclaration Universelle des Droits de l'Homme, 10.12.1948, ONU

DREES, 2011, Comptes nationaux de la santé 2011, s.l., Ministère de l’économie, ministère de la santé, ministère du travail (coll. « Etude et statistiques »)

OMS, 2011, Rapport Mondial sur le Handicap, Genève, Suisse, Organisation Mondiale de la Santé (OMS).

LENGAGNE Pascale, PENNEAU Anne, PICHETTI Sylvain et SERMET Catherine, 2015, L’accès aux soins courants et préventifs des personnes en situation de handicap en France. Tome 1 Résultats de l’enquête Handicap- Santé volet Ménages., s.l., Institut de Recherche et Documentation en Economie de la Santé, Irdes

JACOB Pascal et JOUSSERANDOT Adrien, 2013, Rapport Jacob 2013 : l’accès aux soins et à la santé des personnes handicapées, s.l

PRINCE Martin, PATEL Vikram, SAXENA Shekhar, MAJ Mario, MASELKO Joanna, PHILLIPS Michael R et RAHMAN Atif, 2007, « No health without mental health », The Lancet, septembre 2007, vol. 370, no 9590, p. 859-877

BEATTY Phillip W, HAGGLUND Kristofer J, NERI Melinda T, DHONT Kelley R, CLARK Mary J et HILTON Shelley A, 2003, « Access to health care services among people with chronic or disabling conditions: patterns and predictors 1», Archives of Physical Medicine and Rehabilitation, octobre 2003, vol. 84, no 10, p. 1417-1425


Clemence Bussiere. Recours aux soins de santé primaires des personnes en situation de handicap : analyses économiques a` partir des données de l’enquête Handicap-Santé. Santé publique et épidémiologie. Université Paris-Saclay, 2016. Français.


KANCHERLA Vijaya, VAN NAARDEN BRAUN Kim et YEARGIN-ALLSOPP Marshalyn, 2013, « Dental care among young adults with intellectual disability », Research in Developmental Disabilities, mai 2013, vol. 34, no 5, p. 1630-1641.

LENGAGNE, Pascale, PENNEAU Anne, PICHETTI Sylvain et SERMET Catherine, 2014, L’accès aux soins dentaires, ophtalmologiques et gynécologiques des personnes en situation de handicap en France. Une exploitation de l’enquête Handicap-Santé Ménages, s.l., Institut de Recherche et Documentation en Economie de la Santé, IRDES

ANDRESEN Elena M., PETERSON-BESSE Jana J., KRAHN Gloria L., WALSH Emily S., HORNER- JOHNSON Willi et IEZZONI L I, 2013, « Pap, Mammography, and Clinical Breast Examination Screening Among Women with Disabilities: A Systematic Review », Women’s Health Issues, 1 juillet 2013, vol. 23, no 4, p. e205-e214

HOFFMAN Jeanne M., SHUMWAY-COOK Anne, YORKSTON Kathryn M., CIOL Marcia A., DUDGEON Brian J. et CHAN Leighton, 2007, « Association of mobility limitations with health care satisfaction and use of preventive care: a survey of Medicare beneficiaries », Archives of Physical Medicine and Rehabilitation, mai 2007, vol. 88, no 5, p. 583‐588

RAMIREZ Anthony Farmer, 2005, « Disability and Preventive Cancer Screening: Results from the 2001 California Health Interview Survey », American Journal of Public Health, novembre 2005, vol. 95, no 11, p. 2057-2064

OMS, Rapport sur la santé dans le monde : changer le cours de l'histoire, 2004

PECKHAM Nicholas Guy, 2007, « The Vulnerability and sexual abuse of people with learning disabilities », British Journal of Learning Disabilities, 1 juin 2007, vol. 35, no 2, p. 131-137


Gabbai P. L’avancée en âge des personnes polyhandicapées et infirmes moteurs cérébraux. Journées d’études Polyhandicap. 2010. Comprendre, soigner et accompagner le vieillissement. http://handicap.aphp.fr/files/2012/04/résumé_journée_polyhandicap_2010_AP-HP;pdf


Claude Hamonet, Les personnes en situation de handicap, 8ème édition, 2015

OMS Classification Internationale des Handicaps : Déficiences, Incapacités, Désavantages,Traduction INSERM, Paris, CTNERHI, 1988 (diffusion PUF). 2e édition 1993

F. Chapireau, A. Colvez (1998) Social disadvantage in the international classification of impairments, disabilities and handicaps. Social Science and Medicine, 47,1,59-66.

Pour une étude détaillée et une bibliographie complète dans le domaine de la santé mentale, consulter B. Azéma et coll. (2001), Classification Internationale des Handicaps et santé mentale,coédition CTNERHI GFEP

T.B. Üstün, J.E. Bickenbach, E. Badley, S. Chatterji (1998). A reply to David Pfeiffer, Disability and Society,13,5, p. 830

C. Rossignol (1999). Inadaptation, handicap, Invalidation ?Thèse pour le doctorat d’Etat. Strasbourg

J. E. Bickenbach, S. Chatterji, E. M. Badley, T.B. Üstün, Models of disablement, universalism and the international classification of impairments, disabilities and handicaps, Social Science and Medicine, 48, (1999) 1173-1187.

Saad Z. Nagi, « Some conceptual views in disability and rehabilitation», Sociology and rehabilitation, Ohio State University, 1965

R. Lafon., Vocabulaire de psychopédagogie et de psychiatrie de l’enfant, Paris, Puf, 1963

Clemence Bussiere. Recours aux soins de santé primaires des personnes en situation de handicap : analyses économiques a` partir des données de l’enquête Handicap-Santé. Santé publique et épidémiologie. Université Paris-Saclay, 2016


« le rôle des médecins dans la qualification du handicap », panorama du médecin, n°2739/274/88

OMS Classification Internationale des Handicaps : Déficiences, Incapacités, Désavantages,Traduction INSERM, Paris, CTNERHI, 1988 (diffusion PUF). 2e édition 1993.

B. Veysset, J-P Deremble, Dépendance et vieillissement, Paris, L’Harmattan, 1988

Edgard Morin , La complexité humaine, Paris, Flammarion, 1994, p. 283

http://www.larousse.fr/dictionnaires/francais

Dalla Piazza, S., & Dan, B. (2001). Handicaps et déficiences de l’enfant. In D. B. Université (Ed.), . Bruxelles.


Bornet, C. & Brangier, E. (2013). La méthode des personas : principes, intérêts et limites. Bulletin de psychologie. 
Brangier, E., & Bornet, C. (2011). Persona: A method to produce representations focused on consumers’ needs. In W. Karwowski, M. Soares & N. Stanton (Eds). Human Factors and ergonomics in Consumer Product Design: methods and techniques. 37-61. Taylor and Francis. 
Pruitt, J. S. & Adlin, T. (2006). The persona lifecycle. San Francisco : Morgan Kaufmann. 
Bornet, C., Brangier, E., Deck, Ph., Barcenilla, J., & Bastien, C. (2013). Enrichir la créativité des ingénieurs avec l’analyse de l’activité et les personas : le cas d’un projet d’ergonomie prospective In F. Hubault (Eds). Ergonomie et Société : quelles attentes, quelles réponses ? SELF’2013, Congrès International d’Ergonomie. 11-13 Septembre. Paris, France. 
Le caractère ludique comme levier de performance pour l’anticipation des besoins utilisateurs - Jessy Barré, Stéphanie Buisine, Jérôme Guegan, Fabrice Mantelet, Améziane Aoussat (2014)
Péladeau, P., Romac. B., Rozen, A. & Sevin, C. (2013). L’innovation dans les entreprises en France.Paris : Booz & Company. 

Maguire, M., & Bevan, N. (2002). User requirements analysis. In Usability.Springer US, 133-148. 
Nielsen, L., & Storgaard Hansen, K. (2014). Personas is applicable: a study on the use of personas in Denmark. In Proceedings of the 32nd annual ACM conference on Human factors in computing systems. ACM, 133-148. 
Rosch, E. (1978). Principles of Categorization. In E. Rosch & B. B. Lloyd (Eds.), Cognition and categorization (pp. 27-48). Hillsdale, NJ: Erlbaum 
Blomquist, A., & Arvola, M. (2002). Personas in Action: Ethnography in an Interaction Design Team. In Proceedings of the second Nordic conference on Human-computer interaction (pp. 197-200). ACM 
Cooper, A., Reimann, R., & Cronin, D. (2007). About face 3: the essentials of interaction design. Indianapolis: John Wiley & Sons. 
Brangier, E., Bornet, C., Bastien, J.M.C., Michel, G. & Vivian, R. (2012). Effets des personas et contraintes fonctionnelles sur l’idéation dans la conception d’une bibliothèque numérique. Le Travail Humain, 75(2), 121-145. 
Astbrink, G., & Kadous, W. (2003, January). Using disability scenarios for user centred product design. Proceedings of AATE, Dublin, Ireland. 
Long, F. (2009, May). Real or imaginary: The effectiveness of using personas in product design. In Proceedings of the Irish Ergonomics Society Annual Conference (pp. 1-10). Ireland



Hutmacher, W., Cochrane, D., & Bottani, N. (2001). In pursuit of equity in education: using international indicators to compare equity policies. Dordrecht: Kluwer Academic Publishers.

Friant, N. (2012, November 14). Vers une école plus juste : Entre description, compréhension et gestion du système (Thèse de Doctorat en Sciences Psychologiques et de l’Education). Université de Mons, Mons. Retrieved from http://tel.archives-ouvertes.fr/tel-00752087 

Vedel (G.), "L'égalité" in: La Déclaration des droits de l'homme et du citoyen de 1789. Ses origines, sa pérennité, 1990, p. 172, La Doc. française.

Rivero (J.), "Les notions d'égalité et de discrimination en droit publie français", p. 345, in : Les notions d'égalité et de discrimination en droit interne et en droit international, Travaux de l'Association Henri Capitant, T. XIV, 1961-1962, Dalloz, 1966

Beatrice Plottu. Politiques publiques à incidences sociales: pour une évaluation participative. Économie et Humanisme, 2010, pp.78-82



La France de l'an 2000 (Rapport au Premier Ministre de la commission présidée par Alain Mine), 1994, p. Il, Commissariat général du Plan, Ed. Odile Jacob/La Doc. française; sur ce rapport, v. notamment Matray (L.), "Le rapport Mine sur la France de l'an 2000", Regards sur l'actualité, 1995, nO 207, p. 29 ct s.


Le financement de la protection sociale (Rapport remis au Premier ministre par J.B. de Foucauld), 1995, p. 102, Commissariat général du Plan, La Doc. française ou encore Epiter (J.-P.) et Leteurtre (H.), La protection sociale et son financement, 1995, p. 239, Vuibert


Dictionnaire de l'Académie française, 8ème éd., mot équité.

Dictionnaire Robert, mot équité.

 BAGGOT, R. (1998) Health and Health Care in Britain, 2nd edn. Basingstoke: Macmillan
 

JENSEN Peter M., SAUNDERS Ralph L., THIERER Todd et FRIEDMAN Bruce, 2008, «Factors Associated with Oral Health–Related Quality of Life in Community-Dwelling Elderly Persons with Disabilities », Journal of the American Geriatrics Society, 1 avril 2008, vol. 56, no 4, p. 711-717.


PEZZEMENTI Maureen L. et FISHER Monica A., 2005, « Oral health status of people with intellectual disabilities in the southeastern United States », Journal of the American Dental Association (1939), juillet 2005, vol. 136, no 7, p. 903-912.


ANDRESEN Elena M., PETERSON-BESSE Jana J., KRAHN Gloria L., WALSH Emily S., HORNER- JOHNSON Willi et IEZZONI L I, 2013, « Pap, Mammography, and Clinical Breast Examination Screening Among Women with Disabilities: A Systematic Review », Women’s Health Issues, 1 juillet 2013, vol. 23, no 4, p. e205-e214

TRANI Jean-Francois et LOEB Mitchell, 2012, « Poverty and disability: A vicious circle? Evidence from Afghanistan and Zambia », Journal of International Development, 1 janvier 2012, vol. 24, p. S19-S52

DRESS, les personnes handicapées vieillissantes : une approche à partir de l’enquête HID, n° 204, décembre 2002


Eideliman J.-S., « Aux origines sociales de la culpabilité maternelle. Handicap mental et sentiments parentaux dans la France contemporaine », Revue internationale de l’éducation familiale, n°27, 2010, p. 81-98. 


Caradec V ., « Les transitions biographiques, étapes du vieillissement », Prévenir, n°35, 1998, p. 131-137.

Galland O., « Adolescence, post-adolescence, jeunesse : retour sur quelques interprétations », Revue française de sociologie, vol. 42, n°4, 2001, p. 611-640.

Bidart C., Devenir adulte aujourd’hui. Perspectives internationales, Paris, L’Harmattan, 2006. 

Van de Velde C., Devenir adulte. Sociologie comparée de la jeunesse en Europe, Paris, PUF, 2008.

%pas encore mis dans la biblio

Jean-Sébastien Eideliman. La jeunesse éternelle. Le difficile passage à l’âge adulte des personnes dites handicapées mentales. L’Harmattan. Vivre les âges de la vie. De l’adolescence au grand âge, pp.209-230, 2012, 978-2-296-96212-5 - 2015

Calvez M., « Le handicap comme situation de seuil : éléments pour une sociologie de la liminalité », Sciences sociales et santé, vol. 16, n°1, 1994, p. 61-87

Van Amerongen A-P, Familles et patient souffrant de schizophrénie: questions liées à l’avancée en âge?, Annales medio-psychologiques (2008), doi:10.1016/j.amp.2009.06.007


https://www.researchgate.net/publication/229920687_Mesures_prises_par_Singapour_face_au_vieillissement_aux_inegalites_et_a_la_pauvrete_evaluation

@article {ISSF:ISSF302,
author = {Asher, Mukul G. and Nandy, Amarendu},
title = {Mesures prises par Singapour face au vieillissement, aux inégalités et à la pauvreté: évaluation},
journal = {Revue Internationale de Sécurité Sociale},
volume = {61},
number = {1},
publisher = {Blackwell Publishing Ltd},
issn = {1752-1718},
url = {http://dx.doi.org/10.1111/j.1752-1718.2007.00302.x},
doi = {10.1111/j.1752-1718.2007.00302.x},
pages = {45--66},
year = {2008},
}
