%deuxième chapitre de la partie théorique  : le concept de handicap et le vieillissement
\chapter{La situation de handicap et l'avancée en âge}

Comme on vient de le démontrer dans le chapitre précédent, l'accès aux soins se révèlent particulièrement complexe pour les personnes en situation de handicap mental  Pour réfléchir sur cette problématique, dans cette partie, une réflexion théorique sur le concept de handicap et de la situation de handicap en approfondissant plus particulièrement sur le handicap mental sera présentée.

\section{Le handicap : étude et évolution du concept}
\subsection{Définitions}

Aujourd'hui, le concept de handicap possède des définitions relativement similaires en fonction des organismes.
Par exemple, l'ONU\footnote{Organisation des Nations Unis} définit le handicap comme \og personnes qui présentent des incapacités physiques, mentales, intellectuelles ou sensorielles dont l’interaction avec diverses barrières peut faire obstacle à sur pleine et effective participation sur la base de l’égalité avec les autres\fg (ONU, 2007) \cite{ONU2007}.

D'après l'OMS,\footnote{Organisation Mondiale de la Santé} \og \textit{est handicapée toute personne dont l’intégrité physique ou mentale est passagèrement ou définitivement diminuée, soit congénitalement, soit sous l’effet de l’âge ou d’un accident, en sorte que son autonomie, son aptitude à fréquenter l’école ou à occuper un emploi s’en trouvent compromises}\fg{} (OMS, 2011, \cite{OMS2011}).

Ces deux définitions se rejoignent sur le fait que le handicap est une limitation d'activité et de participation à la société dû à l'environnement dans lequel évolue une personne. Cependant, avant d'admettre cet impact de l'environnement et donc de parler, depuis quelques années, de \og situation de handicap\fg{} (Hamonet, 2015 \cite{Hamonet2015}) , il y a eu beaucoup d'évolution de considération de ce concept et de ces personnes au cours de l'histoire.

\subsection{Évolution de la considération des personnes en situation de handicap à travers l'histoire}


C'est au Moyen-Âge que le concept de handicap se développe et prend de l'importance pour la société. Ce concept est définit comme la capacité à effectuer un travail et à pouvoir subvenir à ses propres besoins. Si une personne ne peut pas faire cela, elle est admise dans des systèmes d’aides. Les personnes les plus représentées sont surtout des vieillards, des veuves, des orphelins, des invalides de guerre. (Ville \& al, 2014 \cite{Ville2014}, p29) Souvent, ces gens sont pauvres et dans des situations précaires. Le fait d’aider ces gens est souvent, pour les personnes riches, source de rachat de ses pêchés auprès de l’Église. Ces personnes handicapées ont donc une forme d’utilité pour la société de l'époque. Il est intéressant de noter qu'à cette époque, le handicap est associé à une supériorité d'une partie de la population, celle pouvant subvenir de manière autonome à ses besoins. (Introduction à la sociologie du handicap)
Le Renaissance et l'avènement du siècle des Lumières (XVIII\textsuperscript{ème}) va considérablement modifier la vision du concept de handicap. On commence à voir apparaître des opportunités et des catégorisation de handicap. Des progrès, surtout grâce à l’éducation, permettent d’envisager la réduction de certains déficit pour des personnes handicapées. De nouvelles notions voient le jour pendant cette période. Plusieurs notions novatrices apparaissent : (Ville \& al, 2014 \cite{Ville2014}, p34)
\begin{itemize}
\item la notion \og d'éducables \fg : personnes dont les déficiences peuvent être compensées par des techniques et des pédagogies adaptées
\item la notion \og d'incurabilité \fg : déficit (mental, physique, sensoriel) irréductible
\item la notion \og d'inadaptation \fg : incapacité qui résulte du déficit et sur laquelle il est possible d’agir
\end{itemize}

Toutes ces notions sont basées sur l’idée que la connaissance n’est pas innée mais provient de notre expérience (Kant, 1781 \cite{Kant1781}). Or, l'expérience se construit grâce à notre interaction et perception (les cinq sens, la proprioception, la mémoire, …) de l’environnement extérieur. S’il y a un déficit quelconque, l'expérience et donc par conséquence la connaissance est altérée, modifiée.
Cependant, cette connaissance n’est plus considérée comme inférieur à celle d’une personne dite \og normale\fg mais différente. Cette différence créé une richesse sociale. C’est la première fois qu’une personne handicapée est considérée comme avantagée dans un domaine. C’est une approche révolutionnaire du déficit qui vaudra à Diderot d’être emprisonné pendant trois mois pour son écrit sur les personnes aveugles (Diderot, 1749 \cite{Diderot1749}) Cependant, les mentalités changent grâce à ce genre réflexion. À titre exemple, c'est en 1760 qu'est créée la première école pour les enfants sourds et muets. Le siècle des lumières met en évidence les différences pour les personnes en situation de handicap et le potentiel, la richesse qu'elles peuvent apporter à la société.

\subsection{Apparition du mot \og handicap\fg{} : étymologie}

La première personne à utiliser ce mot est Samuel Pepys en 1660 (Hamonet, 2015, \cite{Hamonet2015}). Il l'utilise sous la forme \og handicapp \fg{} (avec deux 'p'). Il l’utilise pour désigner une pratique d’échange d’objet de troc. Un arbitre est désigné pour apprécier les différentes valeurs des objets échangés. Pour compenser la différence de valeur, une somme d’argent est déposée dans un couvre-chef. La personne met alors la main dans le chapeau ("the hand in a cap») pour récupérer la somme due. Il s'agit donc d'essayer de réaliser un échange équitable (Sticker et al., 1996 \cite{Sticker1996}). La contraction de cette expression a donné naissance au mot \og handicap \fg{} couramment utilisé de nos jours. 

C'est en Angleterre, au XVII\textsuperscript{ème} siècle que le mot \og handicap\fg{} fait son apparition dans le langage courant avec une application plus particulière pour le monde du hippisme (Hamonet, 2015, \cite{Hamonet2015}).



C'est au XIX\textsuperscript{ème} siècle, en 1827 (en Irlande) qu'apparait le mot \og handicap\fg{} pour la première fois officiellement dans le dictionnaire. Ce terme désigne une pratique devenue courante à l'époque dans les paris équestres. Le but était d'alourdir, au départ de la course, les chevaux jugés supérieurs aux autres afin de ramener l'ensemble des concurrents sur un pied d'égalité et de ramener une part de hasard plus grande dans les paris. Appliqué de cette façon, le handicap est un concept qui permet de ramener l'égalité entre les participants (Borioli \& Laub, 2007 \cite{BorioliLaub2007}; Hamonet, 2015, \cite{Hamonet2015}). On constate dans les deux situations que le terme de handicap désigne le fait de ramener les gens à égalité en imposant un désavantage à une personne avantagé par la situation ou par ses aptitudes.

\subsection{Le handicap s'associe à l'infirmité}
Le XX\textsuperscript{ème} siècle sera aussi une période clef où le concept de handicap va évoluer. Cette évolution est impulsée par deux évènements historiques majeurs que sont la première et la seconde guerre mondiale.

La première guerre mondiale, de 1914 à 1918, laissera, en France, 1 400 000 orphelins et veuves sans revenus et 1 500 000 mutilés de guerre ne pouvant plus travailler (Ville \& al, 2014 \cite{Ville2014} p45). Ces personnes sont des victimes de la guerre qui seront considérés comme des \og handicapés-héros\fg et que la société se devra de dédommager. Les principaux objectifs sont la réadaptation et la réintégration de ces personnes dans la société, qui est à reconstruire. C'est suite à la première guerre mondiale que le mot \og handicapé\fg fera son apparition officielle dans le dictionnaire de la langue française. Il évoque alors la gêne, le désavantage, l'infériorité que supporte une personne par rapport aux conditions normales d'action et d'existence. La vision du handicap est modifiée : l'idée d'écart à la norme est toujours présente mais cette fois-ci, la personne handicapée est en dessous de la norme. Un changement de sémantique se produit : le terme de \og handicap\fg désigne désormais un défaut propre à une personne qui la place en position d’infériorité par rapport à une autre personne. Il ne désigne plus un désavantage que l’on impose à un individu plus, voire trop, supérieur à la norme établit.

La seconde guerre mondiale, au bilan encore plus lourd que la première guerre mondiale, va accentuer cette tendance. De plus, la politique  mise en place en Allemagne par le régime nazi à partir de 1934 impose de stériliser et/ou d'assassiner les personnes handicapées (physiques et/ mentales). La stigmatisation et la notion d'infériorité et d'infirmité s'en trouve renforcée.


\subsection{L'après seconde-guerre mondiale : l'avènement du handicap}
Suite à la seconde guerre mondiale, la vision du handicap va une nouvelle fois évoluer.
La première source de cette évolution est incarnée par un homme : Franklin Delano Roosevelt, président des États-Unis d'Amérique de 1933 à 1945. Il a, entre autre, mis en place la politique du new deal à suite de la crise financière de 1929 et c'est lui qui était le chef des armées lors de la seconde guerre mondiale : il a énormément oeuvré pour la défaite nazi. C'est un homme qui a marqué l'histoire et su mener des politiques efficaces et tout ça ... en fauteuil roulant. Il montre qu'une personne, malgré son handicap, est tout aussi performante qu'une personne valide, si ce n'est plus. Comme l'a dit un banquier de New-York : \og Ses béquilles l'ont hissé au niveau des dieux\fg (Kapsi, 1988 \cite{Kapsi1988}). Il sera aussi surnommé \og le président courage \fg (Kapsi, 1988 \cite{Kapsi1988}). Pour la population des EU, il marque le courage et la combativité. Il focalise l’attention sur ses valeurs humaines, son sens de la stratégie et de la politique. Il donne une nouvelle image de la personne handicapée et va participer à l'ensemble des évolutions législatives qui vont marquer la fin du XX\textsuperscript{ème} siècle et le début du XXI\textsuperscript{ème} siècle qui vont mettre en place une politique d'intégration dans la société et de statut de citoyen à part entière.

\subsection{Recherches et propositions de modèles explicatifs du handicap}

Comme se questionne Chemillier-Gendreau (encore une ref chelou, voir avec Véronique, voir le livre de Hamonet), \og la question est de savoir si le handicap est un concept sanitaire résultant de la constatation d'une déficience physique ou un concept social résultant seulement d'un préjudice économique \fg{}. Autrement dit, le handicap est-il un concept sanitaire (à prendre en compte uniquement du côté des médecins) ou social (à prendre en compte uniquement d'un côté juridique) ? Ce sont là deux visions, \textit{à priori} opposée qui vont alimenter les différents modèles proposés au cours du temps.

\subsubsection{Première taxonomie des handicaps}
 Lors des premières recherches sur les personnes en situation de handicap, le handicap est considéré comme un attribut de la personne. C'est sur ce mode de raisonnement que Lafon propose en 1963 \cite{Lafon1963} une première taxonomie des handicaps comprenant trois degrés : 
\begin{itemize}
\item la personne handicapée léger permettant une indépendance sans aide extérieure et un développement maximum des possibilités
\item la personne handicapée grave ou sérieux ne permettant pas d'acquérir un degré suffisant d'indépendance et ayant besoin de l'aide constance d'une tierce personne pour subvenir à ses besoins, mêmes les plus élémentaires
\item la personne handicapée moyen ou de gravité intermédiaire ayant des chances de réadaptation et permettant, avec une aide spécialisée, de s'intégrer socialement et professionnellement
\end{itemize}
Lafon précise que le handicap est relatif en fonction du pays, du niveau social, du lieu de résidence, de la tolérance ou encore du soutien de la famille.

Dans cette proposition de taxonomie, le handicap est un attribut, une propriété de la personne. L'objectif est de classée la personne dans une case, que lui soit attribué une caractéristiques pour en déduire la démarche d'accompagnement à adopter et les opportunités auxquelles elle peut prétendre. La maladie n'est pas prise en compte.

\subsubsection{Les premiers travaux sur la modélisation du handicap}
Or, comme le précise Hamonet (2015) \cite{Hamonet2015}, la médecine occidentale a pour objectif de localiser l'organe malade et de déterminer le type de liaison. L'analyse est faite sur l'expression de symptômes et la classification de maladie.
Un des premiers chercheurs à faire le lien entre la maladie et les conséquences sur la vie sociale d'une personne est Saad Z. Nagi en 1965 \cite{Nagi1965}. Il propose une relation de cause a effet entre une pathologie, une lésion corporelle, une limitations des capacités fonctionnelles et le handicap : il fait ainsi le lien entre la maladie et les \og disability\fg{}. Ainsi, le lien avec la maladie est établit mais laa vision très médical du handicap est déjà exclusivement présente. 

\subsubsection{Le modèle CIH}
C'est dans cette vision très médicale du handicap que, dans les années soixante-dix, l'OMS prépare la 9\textsuperscript{ème} révision de la Classification Internationale des Maladies. En effet, l'OMS juge le diagnostic médical insuffisant pour décrire les troubles, en particulier les pathologies sur le long terme. L'OMS décide de mettre au point, sous la direction du rhumatologue britannique Philip Wood un \og \textit{manuel de classifications des conséquences des maladies}\fg{}. Cette classification sera adoptée en 1976 par l'OMS. (OMS, 1993).
En 1980, l'OMS publie en version anglaise la CIDIH (Classification Internationale des Déficiences, Incapacités et Handicaps). Cette classification est dénommée \og Classification Internationale des Handicaps (CIH)\fg{}. Dans ce modèle, l'OMS considère le handicap comme un désavantage social pour un individu donné. Ce handicap est le résultat d'une déficience ou d'une incapacité qui limite ou interdit l'accomplissement d'un rôle considéré comme normal, compte tenu de l'âge, du sexe et des facteurs sociaux et culturels. Ce modèle est très critiqué car il aborde le handicap avec une vision purement bio-médical, sans tenir compte de l'influence de l'environnement. C'est un modèle très linéaire de cause à effet comme l'illustre le schéma \ref{CIDIH1980}.

\begin{figure}[htbp]%insertion d'une image : définition d'un environnement figure
\begin{center}
\includegraphics[scale=0.85]{images/SchemaWood.png} 
\end{center}
\caption{Schema de Wood proposé en 1980}
\label{CIDIH1980}
\end{figure}

Les définitions des différents éléments de \ref{CIDIH1980} sont fournis par l'OMS comme suit : 
\begin{itemize}
\item déficience : dans le domaine de la santé, la déficience correspond à toute perte ou altération d'une structure ou fonction psychologique, physiologique ou anatomique
\item incapacités : dans le domaine de la santé, une incapacité correspond à une réduction (résultant d'une déficience) partielle ou totale, de la capacité à accomplir une activité d'une façon ou dans les limites considérées comme normales par un être humain
\item désavantage : dans le domaine de la santé, le désavantage social pour un individu donné résulte d'une déficience ou d'une incapacité qui limite ou interdit l'accomplissement d'un rôle normal (en rapport avec l'âge, le sexe, les facteurs sociaux et culturels)
\end{itemize}

Comme l'ont explicité Chapireau \& Colvez (1998) \cite{ChapireauColvez1998} il est important de noter que ces définitions se limitent au domaine de la santé donc à un domaine relativement restreint concernant le handicap.

Ce modèle a été utilisé dans différents usages (Chapireau \& Colvez, 1998) \cite{ChapireauColvez1998} : 
\begin{itemize}
\item en médecine de la rééducation : cela a permis de faire la différence entre la déficience et l'incapacité en mettant l'accent sur l'importance de l'environnement matériel. Dans ce domaine, ce modèle a prouvé son utilité et sa pertinence
\item en gérontologie : avec ce modèle, les possibilités d'amélioration des déficiences et des incapacités sont faibles voire nulles. Les aides humaines, matérielles et financières sont plus adéquates
\item en rhumatologie : le constat est analogue
\end{itemize}

Ce modèle a aussi été appliqué en santé mentale (Azéma, 2001 --> ref chelou là aussi, voir avec Véro) \cite{RapportAzemaChelou}. Il a permis d'aborder sans clivage les troubles psychopathologiques et leurs conséquences. 

Ce modèle, qui a servi de base de référence internationale pendant près de deux décennies (Üstün \& al, 1998) \cite{Ustun1998}, est exclusivement centré sur un point de vue médical et classificatoire. Il a permis des améliorations dans l'accès aux soins mais seulement pour un nombre limité de domaines. Il est inadapté notamment dans le domaine de l'avancée en âge et du handicap où l'environnement joue un rôle prépondérant mais n'est pas pris en compte dans ce modèle (Rossignol, 1999) \cite {Rossignol1999}. Une révision de ce modèle a été proposée.

\subsubsection{Révision du modèle CIH et proposition d'un modèle plus adapté}

Le principal argument des auteurs proposant la révision de ce modèle (Bickenbach \& al, 1999) \cite{Bickenbach1999} est \og que ce qu’on appelle \og handicap \fg{} physique ou mental n’est pas simplement un attribut de la personne mais une collection complexe d’états, d’activités et de relations, dont beaucoup sont crées par l’environnement social\fg{}. En d'autre terme, l'aspect centré uniquement sur l'axe médical est critiqué. Une ouverture prenant en compte de nouveaux éléments dont l'environnement extérieure et l'environnement social sont important à prendre en compte. De plus, ils ne doivent pas être considérés comme une somme d'éléments mais être intégré à un modèle systémique global.

En conséquence, en 2001, l'OMS a révisé son modèle et en a proposé une nouvelle version en tenant compte notamment de l'influence de l'environnement social et de son influence sur la personne. L'OMS a proposé ce modèle sous l'intitulé \og Classification Internationale Internationale du Fonctionnement, du Handicap et de la Santé\fg{} (CIF). Cette nouvelle version propose des définitions nouvelles et complémentaires par rapport à l'ancien modèle (OMS, 2001) \cite{OMS2001} :

\begin{itemize}
\item les fonctions organiques : fonction physiologiques des systèmes organiques (y compris les fonctions physiologiques) 
\item les structures anatomiques : parties anatomiques du corps telles que les organes, les membres et leurs composantes
\item les déficiences : problèmes dans la fonction organique ou la structure anatomique tels qu'un écart ou une perte importante
\item activité : exécution d'une tâche par une personne
\item participation : implication d'une personne dans une situation de vie réelle
\item limitation d'activité : difficultés que rencontrent une personne dans l'exécution de certaines activités
\item restriction de participation : problèmes qu'une personne peut rencontrer en s'impliquant dans une situation de vie réelle
\item facteur environnementaux : environnement physique, social et attitudinal dans lequel les gens vivent et mènent leur vie
\newline
\end{itemize}


Le premier constat est que, dans toutes les définitions proposées, l'aspect médical, santé n'est plus présent. Ces définitions sont plus larges et prennent en compte une spectre de situation plus important et plus diversifié.

C'est ce modèle, illustré \ref{OMS2001}, qui permet de passer de statut de \og personne handicapée\fg{} au statut de \og personne en situation de handicap\fg{}. En effet, ce modèle considère désormais le handicap comme une condition humaine universelle, une variation dans le fonctionnement humain. Le terme est redéfinit comme \og une restriction de la vie sociale résultant de l’interaction entre une limitation d’activité, consécutive à un problème de santé et des obstacles environnementaux.\fg{} L'environnement extérieur et par extension, la situation, le contexte dans lequel se trouve la personne prend donc toute son importance. Le modèle proposé est un modèle systémique dont chaque composante influe sur l'autre : il n'y a plus de linéarité entre les éléments, c'est un modèle systémique. 

\begin{figure}[htbp]%insertion d'une image : définition d'un environnement figure
\begin{center}
\includegraphics[scale=0.99]{images/CIF_OMS2001.jpg} 
\end{center}
\caption{Modèle CIF - OMS - 2001}
\label{ModeleCIF2001}
\end{figure}

\subsubsection{Modèle de Processus de Production du Handicap proposé par Fougeyrollas}

En 1998, un chercheur canadien, Patrick Fougeyrollas \cite{Fougeyrollas1998}, propose un nouveau modèle plus complet : \og le processus de production des handicaps\fg{}. C'est un modèle systémique global et interactif qui prend en compte aussi les environnements extérieurs. La nouveauté en distinguant les facteurs personnels propres à chaque personne (capacités cognitives, perceptives, habitude de vie ...) et les facteurs environnementaux (présence d'obstacles ou de facilitateurs, ...)

La grande nouveauté réside dans la prise en compte des facteurs contextuels et la différenciation, au sein de cette catégorie, des facteurs environnementaux (qui représentent les éléments extérieurs et que la personne ne maîtrise pas) et des facteurs personnels (qui représentent les éléments culturels, les habitudes de vie, ...). Par exemple, la médication représente la seconde causes des situations de handicap, après les effets de l'âge (Hamonet, 2015) \cite{Hamonet2015}. Une mauvaise médication peut être due à des difficultés de lisibilités sur la boite de médicament ou à une ordonnance mal écrite et mal interprétée : cela dépend des facteurs environnementaux. Cependant, il se peut aussi que la mauvaise médication soit due à une non adhésion à un traitement et que cela engendre des effets secondaires néfastes. Auquel cas, ce sont les facteurs personnels qui influent. Ce modèle permet de prendre en compte ces deux éléments dans le processus de production du handicap.
 


\begin{figure}[htbp]
\begin{center}
\includegraphics[scale=0.75]{images/PPH.jpg} 
\end{center}
\caption{\label{PPH}Modèle de Processus de Production du Handicap (PPH) proposé par Fougeyrollas en 1998} 
\end{figure}

\subsection{Les principales lois relatives aux personnes en situation de handicap}
\subsubsection{La loi de 1975}
Une étape législative importante, en France, de cette période est l'année 1975 et le vote de la loi \og orientation en faveur des personnes handicapées\fg \cite{Didier2005}. L'objectif de cette loi est d'harmoniser les traitements disponibles, améliorer et favoriser l'intégration sociale et professionnelle des personnes handicapés et de permettre un retour à une vie \og ordinaire\fg . Ce changement de vision au niveau législatif est notamment impulsé par la ministre de la santé de l'époque, Madame Simone Veil, qui considère qu'il ne faut plus considérer les origines ou les causes du handicap mais les conséquences de celui-ci sur la vie de la personne concernée et son interaction avec l'environnement extérieur. Cette nouvelle manière de penser influe grandement sur la sémantique : le terme de \og personne en situation de handicap\fg fait son apparition au détriment du terme \og personne handicapée\fg . Il y a une prise de conscience du handicap et de sa relation avec l'environnement extérieur, de la situation étudiée et des aides à mettre en place. En effet, c'est la loi numéro 535 qui permet la création des COTOREP (Comission Technique d'Orientation et de Reclassement Professsionnel), aujourd'hui remplacé par les ESAT. Cette loi permet également la création de l'Allocation aux Adultes Handicapées (AAH). 
Cette même année, le 9 Décembre, à l'ONU est voté la résolution du \og Droits des Personnes Handicapées\fg qui propose de mettre en place un régime de protection social dédié à ces personnes. Cette résolution définit l’accessibilité comme l'aménagement de l’environnement et des mesures ponctuelles en cas de présence effective régulière d’un utilisateur en fonction de son handicap. Il n'y a cependant pas de définition de délai et de sanction en cas de manquement à ces règles. En 1981, l'ONU déclarera l'année \og année de la personne handicapée\fg , preuve de la prise de conscience de cette partie de la population et de l'ouverture d'esprit qui s'initie par rapport à celle ci.

\subsection{La loi du 2 janvier 2002}
Le 2 janvier 2002 \cite{Levy2002}, le gouvernement vote une loi importante : celle-ci rénove l'action sociale et médico-sociale par rapport à l'accompagnement de ces personnes. Elle favorise notamment la diversification des dispositifs d'accueil et d'accompagnement. 
Un article est particulièrement intéressant : l'article 53. Il introduit notamment la notion de droit à la compensation pour ces personnes. Quelque soit l'âge, le sexe et le type de handicap, la personne a le droit de recevoir wune allocation pour subvenir à ses besoins vitaux. 

\subsubsection{11 février 2005 : adoption de la loi numéro 102}

Le 11 février 2005, la France vote et adopte la loi numéro 102 \cite{loi2005.102} \og pour l'égalité des droits et des chances, la participation et la citoyenneté des personnes handicapées\fg . Cette loi marque une avancée importante dans la considération du handicap et l'intégration professionnelle et sociale des personnes en situation de handicap. 

Cette loi définit le handicap comme \og limitation d’activité ou restriction de participation à la vie ne société subie dans son environnement par une personne en raison d’une altération substantielle, durable ou définitive d’une ou plusieurs fonctions physiques, sensorielles, mentales, cognitives ou psychique, d’un polyhandicap ou d’un trouble de santé invalidant\fg . 
Cette loi propose une changement de référentiel significatif dans la définition du handicap : il n’y a plus d’écart à la norme, à un homme moyen définit statistiquement comme \og valide\fg. Le nouveau référentiel est l’humanité dans son ensemble hétérogène, sa diversité et la prise en compte de l’environnement extérieur.
La notion de droit à la compensation est accentuée et complétée : elle est désormais considérée comme droit universel accordé à toute personne en situation de handicap pour satisfaire ses besoins personnels et l’aider à l’élaboration de son projet de vie. La nouveauté résidé dans l'aide à l'élaboration du projet de vie : cette compensation n'est plus là uniquement pour subvenir aux besoins vitaux mais pour considérer la personne comme un être à part entière avec ses ambitions et ses projets. Cette aide est là aussi pour l'aider dans ces aspects de sa vie :



\begin{itemize}
\item une participation à la vie de la société des personnes en situation de handicap pour participer aux institutions c'est-à-dire une considération de la personne comme un citoyen à part entière de la société
\item un service adapté et spécifique à la personne autrement dit l'adaptation de l'environnement à la personne et non plus l'inverse
\item un droit à la compensation complet pour subvenir aux besoins et aux projets de vie de la personne
\end{itemize}

Le statut passe de \og personne handicapée\fg{} à une catégorie sociale et une catégorie de recherche.

Un autre élément important de cette loi est l'apparition du terme de \og handicap invisible\fg{} dont Le handicap cognitif, intellectuel ou mental. Cette population est particulière vulnérable. Sa considération marque donc une avancée importante dans l'amélioration de l'accès aux soins pour eux. Il convient cependant de s'attarder plus longuement sur ce concept de handicap mental.


\subsection{Le handicap mental}
Dans cette partie, le concept de handicap mental va être abordé de manière plus précise.

\subsubsection{Définitions}

Le handicap est un concept relativement hétérogène auquel on peut associer plusieurs définitions différentes.
Une première définition du handicap mental peut être la suivante : le handicap correspond à \og des perturbations du degré de développement des fonctions cognitives telles que la perception, l’attention, la mémoire et la pensée ainsi que leur détérioration à la suite d’un processus pathologique\fg (Dalla Piazza \& al, 2001) \cite{Piazza2001}. La définition proposée par ces auteurs repose essentiellement sur des éléments scientifiques, médicaux : en effet, ils raisonnent uniquement en terme de capacités perceptives et cognitives et des conséquences de ces impacts sur la personne d'un point de vue pathologique donc médical.
Une deuxième vision de ce concept est que le handicap mental est associé à \og une personne ayant une déficience intellectuelle avec une capacité plus limité d’apprentissage et un développement de l’intelligence qui diffère de la moyenne des gens\fg . (OMS, 2001) \cite{world2001classification}. Ici, les auteurs se référent à une norme sociale, à un écart par rapport aux autres personnes de la société.
Enfin, de manière plus synthétique,  une autre vision supplémentaire du handicap mental qui peut être proposée est que la handicap mental est \og une conséquence sociale d'une déficience intellectuelle\fg (UNAPEI, 2013) \cite{UNAPEI2013} (voir avec Véro par rapport à la ref : est ce que je met les auteurs du rapport ou est-ce que je mets l'UNAPEI ?). Une personne qui a une déficience intellectuelle est une personne qui a des capacités d'apprentissage et de développement de l'intelligence limité. Cela correspond à un ensemble d’incapacité du système nerveux qui a des conséquences sur les fonctions cognitives et par extension, d'après l'UNAPEI, sur la situation sociale de la personne. On retrouve ici un aspect non évoqué dans les définitions précédentes : l'aspect social donc de l'environnement extérieur.


Pour synthétiser par rapport à ces différentes définitions proposées, le handicap mental qualifie à la fois une déficience intellectuelle (aspect scientifique, médical) et les conséquences au quotidien de cette déficience (aspect social). Le handicap mental se traduit par des difficultés plus ou moins importantes de réflexion, de conceptualisation, de communication ou encore de prise de décision dans un contexte donné. Dans la suite de ce document, le concept abordé sera celui de handicap mental car, comme le souligne Eideliman (2010) \cite{Eideliman2010}, il faut considérer le concept de \og handicap mental \fg{} et non pas de \og déficience intellectuelle \fg{} ou de \og retard mental \fg{} car le handicap mental englobe mieux le fait qu’il s’agisse d’une construction sociale, d’une entité saturée d’enjeux sociaux, politiques et moraux. En utilisant donc le terme de \og handicap mental \fg{}, cette étude se place dans la considération de la loi de 2005, approche pertinente.


\subsubsection{Classification}

Pour ordonner et classer les situations de handicap mental, le QI\footnote{Quotient Intellectuelle} est utilisé. Le retard mental est caractérisé par un faible QI : un QI inférieur à 70 (OMS, 1988) \cite{OMS1988}. La répartition du QI d'une population se fait selon une distribution gaussienne avec une moyenne de 100 et un écart type de 15.
Dans une courbe gaussienne, les valeurs comprises en dessous de la valeur égale à la différence entre la moyenne et deux fois la valeur de l'écart type représente un pourcent de la population étudié.
Cela signifie donc que 1 pourcent de la population a un QI inférieur ou égal à 70 donc, par conséquent, qu'un pourcent de la population est en situation de handicap mental.
La classification proposée est la suivante (OMS, 1988) \cite{OMS1988} :
\begin{itemize}
\item QI < 20 : retard mental profond : cela correspond au développement intellectuel d’un enfant de moins de 3 ans
\item 20 < QI < 34 : retard mental sévère : cela correspond au développement intellectuel d’un enfant entre 3 et 6 ans
\item 35 < QI < 49 : retard mental moyen : cela correspond au développement intellectuel d’un enfant entre 6 et 9 ans
\item 50 < QI < 70 : retard mental léger : cela correspond au développement intellectuel enfant de 10 ans
\end{itemize}

\subsubsection{Une population de plus en plus représentée}

Aujourd'hui, cette population voit son espérance de vie augmenté avec une tendance encore plus marquée que la population valide. En effet, en 1930, pour une personne en situation de handicap mental, l'espérance de vie était de 19,9 ans pour un homme et de 22 ans pour les femmes. En 1980, l'espérance de vie était respectivement de 58,3 ans et de 59,3 ans (Carter \& Jancar, 1983) \cite{CarterJancar1983}. Malgré cette nette amélioration de leur espérance de vie, ces personnes subissent un vieillissement précoce comme l'expliquent Azéma \& Martinez (2005) \cite{AzemaMartinez2005} \og il existe un effet cumulatif des troubles dégénératifs liés à l’âge avec les incapacités préexistantes. Les maladies chroniques invalidantes survenant lors du processus de vieillissement normal, viennent \og ajouter de l’incapacité à l’incapacité\fg{} \fg{} . Une autre étude de la fondation John Bost (1991 (ref trouvé dans le livret blanc de l'UNAPEI page 13, voir avec Véro pour la gestion de la ref) \cite{Gabbai2010} confirme cette tendance : pour toutes les catégories de handicap, entre 1972 et 1979, l'espérance de vie était de 48 ans. Pour les personnes nées entre 1980 et 1990, elle était de 60 ans, soit une augmentation de 12 ans. Cette même étude montre également que l'allongement de l'espérance de vie globale de la population générale bénéficie en grande partie aux personnes en situation de handicap mental, aux personnes déficientes moyens et légers. Ces études confirment les prévision de Lenoir (1974) \cite{Lenoir1974} \og Les débiles profonds mouraient presque tous à l’adolescence. Ils atteignent maintenant l’âge mur et nous aurons, dans 10 à 15 ans, de grands handicapés du 3ème âge\fg . Cette tendance se confirme et le nombre de personne vivant en incapacité s'accroit (Azéma \& Martinez, 2005) \cite{AzemaMartinez2005}. Les données épidémiologiques restent assez faibles par rapport à la présence de ces personnes dans nos sociétés. Il est tout de même possible de citer cette étude menée par la DREES en 2002 \cite{Drees2002}. Elle se base sur la définition de personne handicapée vieillisante suivante : \og personne de 40 ans ou plus qui présente au moins une déficience survenue avant l'âge adulte et une incapacité survenue avant l'âge de 20 ans\fg{}. Sur cette base, l'étude spécifie, qu'en France, il y a 635 000 personnes (environ 1 \% de la population) en situation de handicap avançant en âge dont 267 000 ont plus de 65 ans.
Un nouveau type de public apparaît donc dans la société : des personnes handicapées du 3\textsuperscript{ème} âge ou encore des personnes handicapées âgées. 
Le vieillissement de ces personnes impliquent l'apparition de nouveaux problèmes (Azéma \& Martinez, 2005) \cite{AzemaMartinez2005}: 
\begin{itemize}
\item risques de développement des maladies neuro-dégénératives plus fréquents. Par exemple, des personnes atteintes du syndrome de Down (i.e. trisomie 21) développe des premières démences dès l"âge de 30 ans
\item risques de développement d'un cancer plus élevé
\item risques de maladie cardio-vasculaire plus élevés
\end{itemize}

Pour définir ces personnes avec précision, il convient d'abord de réfléchir sur cette notion de vieillissement, de personne âgées.

\section{Le vieillissement et l'avancée en âge}


\subsection{Définition du vieillissement}

Plusieurs définitions peuvent être proposées par rapport à ce concept de vieillissement.
La première définition qui peut être retenue est que le vieillissement est un phénomène normal, progressif, irréversible, inégal, hétérogène et ayant des conséquences délétères sur la personne (Gohet, 2013) \cite{Gohet2013}. Cette définition marque d'une part l'importance du temps (puisque c'est un phénomène progressif et irréversible) : on peut retarder le vieillissement mais on ne peut pas y échapper. En effet, le vieillissement est un phénomène qui débute dès la naissance (CNSA, 2010) \cite{CNSA2010}. D'autre part, la définition étudiée est péjorative : les effets du vieillissement sont uniquement délétères. EN effet, le vieillissement permet, entre autre, d'acquérir de l'expérience. Cette expérience permet d'être plus critique vis-à-vis de certaines situations ou problématiques auxquelles peut faire face une personne. L'OMS propose une définition alternative mais relative similaire à celle-ci. Le vieillissement est un \og processus graduel et irréversible de modification des structures et des fonctions de l'organisme résultat du passage du temps\fg{}. La grande différence avec la définition proposée par Gohet réside dans le terme de \og modifications \fg{}. En effet, une modification peut être délétère ou positive. Par exemple, la diminution de l'acuité visuelle d'une personne (donc la modification des capacités visuelles de la personne) est un effet délétère. En revanche, le fait de se souvenir de situations particulières (donc d'avoir une modification de l'expérience) peut être bénéfique pour adapter les décisions à prendre.  


Néanmoins, la définition proposée par Gohet met en valeur la variabilité intra-indidivuelle de ce phénomène : il est propre à chaque personne. Deux personnes ne vieillissent pas de la même façon et les effets ne sont pas les mêmes en fonction de chaque individu. Cependant, cette notion ne permet pas de définir d'âge ou de stade au processus de vieillissement. Comme le souligne Eideliman, \og il est des cas particuliers où l'évidence de l'avancée en âge est fortement remise en cause et devient un enjeu de définition voire de conflit\fg{}


Il est d'ailleurs de plus en plus difficile, à l'heure actuelle, de définir un rapport entre l'âge et le vieillissement comme le souligne Brami (2014) \cite{Brami2014} : \og s’il y a longtemps eu cohérence entre l’âge biologique et l’âge sociale, aujourd’hui il y a dissociation : on est socialement vieux de plus en plus jeune et biologiquement vieux de plus en plus tard\fg . Selon Bourdelais (1993) \cite{Bourdelais1993}, il faut attendre 75-80 ans pour ressembler en terme de santé aux séxagénaires des années soixante. On en déduit que le vieillissement est un phénomène dynamique.
En effet, d'après Montaigne (XVI\textsuperscript{ème} siècle, la vieillesse apparait à l'âge de 30 ans. Au XVII\textsuperscript{ème} siècle, c'est autour de 40 ans. 1950, on évoque la vieillesse à partir de 60 ans et à partir de 65 dans les années 2000. En 2060, on estimera qu'on pourra considérer une personne comme vieille à partir de 75 ans. Ce dynamisme et cette réflexion par rapport à l'âge absolue pour évaluer le vieillissement montre la complexité de définition et surtout le fait que c'est un concept nouveau mal maitrisé à l'heure actuelle. \og Le concept de vieux est un nouveau-né\fg précisait déjà Cyrulnik il y a plus de vingt ans en 1989 \cite{Cyrulnik1989}. Ce constat est désormais encore plus vrai. Aujourd'hui, la loi définit une personne comme étant âgé à partir de 60 ans. Cependant, \og le vieillissement en tant que tel et l’âge d’une personne ne permettent en aucun cas d’en déduire ses besoins\fg (CNSA, 2010) \cite{CNSA2010}.

Le concept de vieillissement est donc un concept dynamique extrêmement difficile à définir et qui ne permet pas d'apporter des réponses pertinentes et adaptées à une personne. Le vieillissement est propre à chaque individu et deux individus qui ont le même âge n'ont pas la même \og vieillesse \fg{}.
La loi française, elle-même, va dans ce sens. Selon l'article 489 du code civil, pour les personnes en situation de handicap mental le passage de l'enfance à l'âge adulte est fixé à 20 ans. Contre 18 ans pour le reste de la population. Cela signifie qu'une personne en situation de handicap mental de 20 ans et une personne valide de 18 ans n'ont pas le même âge mais, paradoxalement, ont la même vieillesse. Aborder le vieillissement uniquement avec comme critère l'âge de la personne n'est pas pertinent. De plus, cela suggère sur la seule échelle disponible pour mesurer les décalages est l’âge.
Eideliman résume le problème avec formulation suivant : \og Alors que l’accès à la majorité est marquées pour les individus dits normaux par un certain nombre de rites de passage et l’accès à de nouveaux droits, il est largement invisibilisé pour les personnes en situation de handicap mental\fg{}

Par conséquent, il est plus judicieux d'aborder cette problématique sous l'angle de l'avancée en âge.

\subsection{L'avancée en âge}

La CNSA\footnote{Caisse Nationale de Solidarité Autonomie} (2010) \cite{CNSA2010} définit l'avancée en âge comme le passage d'un âge à un autre, qui peut parfois être critique. 
De manière assez unanime, les sociologues ont proposé plusieurs modèles décrivant les étapes socialement construites qui rythment l'avancée en âge dans les sociétés occidentale contemporaines. Les étapes qui reviennent, chronologiquement, sont l'enfance, l'adolescence, l'âge adulte et la vieillesse (Caradec, 1998 \cite{Caradec1998}; Galland, 2001 \cite{Galland2001}; Bidard, 2006 \cite{Bidard2006}; Van De Velde, 2008 \cite{VanDeVelde2008}). Pour des jeunes de 10 à 20 ans, en situation de handicap mental, il est difficile de désigner leur appartenance d'âge (Eideliman, 2015). De plus, pour différentes raisons, pour les personnes en situation de handicap mental, le passage vers l'âge adulte (qu'il est possible de lier à des critères comme l'activité professionnelle, l'indépendance, ...) est retardée voire impossible (Eideliman, 2015). Pour cette population, la définition de jeune est difficile à définir. Il faut ajouter à cela un passage à l'âge adulte qui est  complexe voire impossible. Avec ces considérations, est-il possible de parler d'étapes de la vie (comme la jeunesse, l'âge adulte ou la vieillesse) pour cette catégorie de la population ? Calvez propose une approche différente de cette évolution avec le concept de liminalité. Selon lui, le passage d'un âge à un autre pour une personne en situation de handicap mental est difficile à définir. C'est une situation d'entre deux par rapport au développement habituel des autres enfants. Le concept de liminalité permet d'approcher cette différence de développement. Cette mesure de différence de développement (donc la liminalité) permet de mesurer les différences d'avancée en âge entre deux individus. 


Ce concept permet donc d'intégrer le dynamisme dans la notion de vieillissement et de définir le vieillissement comme un seuil de criticité provoqué par le passage d'un âge à un autre. Cette criticité peut être mesurée par des données basées sur les capacités physiques, cognitives ou encore psychique tel que l'espérance de vie. Par exemple, à 40 ans, l'espérance de vie sans incapacité est de 32 ans pour un ouvrier et de 40 ans ans pour un cadre (INSEE, 2011) \cite{INSEE2011}. Cela signifie qu'au même âge, l'ouvrier est déjà plus vieux de 8 ans qu'un cadre. L'avancée en âge, donc le vieillissement par extension, est plus rapide chez un ouvrier que chez un cadre car le seuil de criticité est plus élevé chez un ouvrier du fait du mode de vie et des conditions de travail. 

La considération du concept \og d'avancée en âge\fg{} permet également d'éviter les changements de catégories pour les personnes et les différences de traitement qui en découlent. Car, comme le remarque Van Amerongen (2008) \og Le malade schizophrène reste un malade psychiatrique et ne devient pas un malade gériatrique.\fg{}. Or, Bruggeman insiste sur le fait \og La vieillesse et le handicap font l’objet de deux politiques publiques bien distinctes, entraînant des prises en charges différentielles, en particulier au niveau des institutions \fg{} ce qui est absurde, comme cela vient d'être démontré. 

\section{Réflexion sur la dépendance, la fragilité et le handicap}

\subsection{Étude du concept de dépendance}

Comme l'explique Veysset \& Deremble (1988) \cite{VeyssetDeremble1988}, le mot dépendance vient du latin \og dependere \fg{} qui signifie pendre, suspendre. Il est possible d'y voir deux sens : 
\begin{itemize}
\item un premier sens plutôt négatif qui fait référence à la soumission d'une personne à une autre
\item un deuxième sens plutôt positif qui fait référence à l'échange entre différents partis, que c'est un fait de la vie sociale et que la vie en société est faite d'interdépendance
\end{itemize}

Toujours selon Veysset \& Deremble (1988) \cite{VeyssetDeremble1988}, ce terme de dépendance à utiliser à chaque fois qu'une personne dépend d'une intervention technique extérieure ou d'une aide partielle ou totale d'une autre personne. 

\subsection{Étude du concept d'autonomie}
Le mot autonomie vient du grec \og \textit{autos}\fg{} qui signifie \og \textit{soi-même}\fg{} et du mot \og \textit{nomeil}\fg{} qui signifie \og \textit{gouverner}\fg{}. L'autonomie signifie \og \textit{se gouverner soi-même}\fg{}. 

Il apparaît donc que les concepts d'autonomie et de dépendance sont incompatibles. En effet, comment peut-on se gouverner soi-même (donc être autonome) si nous avons besoins des autres pour réaliser nos actions (donc être en situation de dépendance) ? Comme le précise Hamonet (2015) \cite{Hamonet2015}, on oppose souvent l'autonomie à la dépendance et on assimile la dépendance au handicap. 

Cependant, comme l'explique Edgard Morin en 1994 \cite{Morin1994} : \og \textit{Toute autonomie se construit par et dans la dépendance écologique}\fg{}. Le terme  \og \textit{écologique}\fg{} se définit comme relation entre les êtres vivants dans leur habitat et leur environnement ainsi qu'avec les autres êtres vivants, même s'ils ne sont pas de la même espèce. Ainsi la dépendance implique et mène à l'autonomie. La réciproque n'est pas vrai. En conséquence, les termes d'autonomie, de dépendance et de handicap ne sont pas incompatibles mais, au contraire, complémentaires les uns des autres.

\subsection{Autonomie, dépendance, handicap}


